\documentclass[../00_main.tex]{subfiles}

\begin{document}

\subsection{Problem Functions}

All problem functions in this report are versions of the following general
equation
\begin{equation}\label{eq:base}
    \begin{cases}
        \epsilon \cdot u''(x) + a(x) \cdot u'(x) - b(x) \cdot u(x) &= f(x),
            \quad x \in (0,1) \\
        \zeta_0 \cdot u(0) - \eta_0\cdot\epsilon\cdot u'(0) &= \phi_0,   \\
        \zeta_1 \cdot u(1) + \eta_1\cdot\epsilon\cdot u'(1) &= \phi_1.
    \end{cases}
\end{equation}
Here $\epsilon > 0$, $a(x), b(x), f(x)$ are functions on the interval $[0;1]$
where $b(x) \ge 0$. The following problems are versions of the general problem 
\eqref{eq:base}. The first equation from my readme file is the following:
\begin{equation}\label{eq:bvp1}
    \begin{cases}
        \epsilon \cdot u''(x) + u'(x) &= x^3,
            \quad x \in (0,1) \\
        u(0) &= \phi_0,   \\
        u(1) &= \phi_1.
    \end{cases}
\end{equation}
Here $\phi_0$ and $\phi_1$ can be freely chosen. I found the analytic 
solution to this equation for our first homework assignment. 
\begin{equation}\nonumber
    \begin{gathered}
        u = (e^{-1/\epsilon\cdot x} - 1) 
            \frac{\phi_0 - \phi_1 + 1/4 - \epsilon+3 \epsilon^2
            - 6 \epsilon^3} {1 - e^{-1/\epsilon}}\\
         + \frac{1}{4}x^4-\epsilon x^3+3\epsilon^2 x^2-6\epsilon^3 x+\phi_0
    \end{gathered}
\end{equation}
The boundary conditions of this problem make it a Dirichlet problem. I will
refer to this problem as \textsc{Problem 1} from now on. The second equation 
from my readme file is:
\begin{equation}\label{eq:bvp2}
    \begin{cases}
        \epsilon \cdot u''(x) + u'(x) &= x^3,
            \quad x \in (0,1) \\
        u(0) - u'(0) &= \phi_0,   \\
        u(1) &= \phi_1.
    \end{cases}
\end{equation}
This equation's analytic solution is:
\begin{equation}\nonumber
    \begin{gathered}
        u = \phi_0  + (e^{-1/\epsilon \cdot x} - 2)
            \frac{\phi_0-\phi_1+1/4 -\epsilon+3\epsilon^2-6\epsilon^3-
            6\epsilon^4}{2 - e^{-1/\epsilon}}    \\
        + \frac{1}{4}x^4 - \epsilon x^3 + 3\epsilon^2 x^2
            - 6 \epsilon^3 x - 6 \epsilon^4.
    \end{gathered}
\end{equation}
Again $\phi_0$ and $\phi_1$ can be freely chosen. This problem is a Robin 
problem because it contains a derivative in its boundary condition. This problem 
will be referred to as \textsc{Problem 2}. The third problem this report covers 
is number 3 from our PDF. It has the form:
\begin{equation}\label{eq:bvp3}
    \begin{cases}
        \epsilon \cdot u''(x) + \left(3\cdot(1+x)^2-\frac{2\cdot\epsilon}{1+x}
            \right) \cdot u'(x) &= \frac{3\cdot\epsilon}{2\cdot(1+x)^2} 
            - \frac{3\dot(1+x)}{2},
            \quad x \in (0,1) \\
        u(0) - \frac{1}{3}\cdot\epsilon\cdot u'(0) &= \frac{\epsilon}{6}
        - \frac{1}{1-e^{-7/\epsilon}},   \\
        u(1) &= 1 - \frac{\ln(2)}{2},
    \end{cases}
\end{equation}
and it's analytic solution is 
\begin{equation}\nonumber
    u(x) = \frac{1 - e^{\frac{1-(1+x)^3}{\epsilon}}}{1-e^{-7/\epsilon}} 
        - \frac{\ln(1+x)}{2}.
\end{equation}
This problem is also a Robin problem and will be referred to as 
\textsc{Problem 3}.

\subsection{Finite Difference Schemes}

There are two finite difference schemes that my program is capable of and
this report covers. The first scheme is the Central Difference Scheme (CDS) for 
which
\begin{equation}\nonumber
    \begin{gathered}
        \gamma_i = 1,  \\
        \theta_i = 0,  \\
    \end{gathered}
\end{equation}
and the Ilin Scheme (IS) where
\begin{equation}\nonumber
    \begin{gathered}
        \gamma_i = R_i \cdot \text{cotanh}(R_i),        \\
        \theta_i =\text{cotanh}(R_i) - \frac{1}{R_i}.   \\
    \end{gathered}
\end{equation}
This is the only difference between IS and the CDS. They both use the Thomas 
Algorithm with the values described in our task PDF to solve the equations 
\eqref{eq:bvp1}, \eqref{eq:bvp2}, and \eqref{eq:bvp3} numerically. For 
the Thomas Algorithm the factors $A_i, B_i, C_i$ from our advection-diffusion
PDF p. 20 are used.
My boundary conditions are based on the precise boundary conditions that are
outlined in lab 1 PDF on page 3 (IV.1.3). These precise conditions allow
precise approximations of the boundaries and by extension the whole function.

\end{document}
