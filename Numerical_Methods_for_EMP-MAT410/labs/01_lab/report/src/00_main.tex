\documentclass[a4paper, 12pt, reqno]{article}

\newcommand{\titl}{NumMeth MAT-410 Lab 1 Report}
\newcommand{\auth}{Moritz M. Konarski}
\usepackage{amsmath}
\usepackage{amssymb}
\usepackage[margin=1in]{geometry}
\usepackage[english]{babel}
\usepackage[pdftitle={\titl}, pdfauthor={\auth}, final]{hyperref}
\usepackage{graphicx}
\usepackage{mathptmx}
\usepackage[T1]{fontenc}
\usepackage{accents}
\usepackage{tikz}
\usepackage{float}
\usepackage{pgfplots}
\pgfplotsset{compat=1.17}
\usepackage{pgfplotstable}
\newcommand{\figref}[1]{Figure~\ref{#1}}
\newcommand{\tabref}[1]{Table~\ref{#1}}
\renewcommand{\baselinestretch}{1.3}
\setcounter{tocdepth}{2}
\usepackage{accents}
\usepackage[document]{ragged2e}

\setlength{\RaggedRightParindent}{0.25in}

\usepackage{subfiles}

\title{\titl}
\author{\auth}
\date{\today}

\begin{document}
\maketitle
\begin{abstract}\noindent
    This report deals with numerical solutions to second order differential
    equations using finite difference methods. The problems were chosen 
    according to my readme file. The finite difference methods in this report 
    are the Central Difference Scheme and the Ilin Scheme. With their help the 
    test problems' analytical and numerical solutions for various parameter 
    values are considered.
\end{abstract}
\tableofcontents

\section{Introduction}

This report deals with three differential equations based on one general system 
of equations. These equations are solved both analytically and numerically.
The numerical solutions are found using finite difference methods. The 
resulting graphs of the solutions for various parameters are shown. Their 
behavior is investigated according to the requirements in our task PDF.

\subfile{../src/01_intro}

\section{Analysis}

This section will investigate the behavior of the analytical and numerical
solutions to \eqref{eq:bvp1}, \eqref{eq:bvp2}, and \eqref{eq:bvp3}. The 
parameters $n$ (number of nodes) and $\epsilon$ will be changed to investigate
their influence on the problem functions and solutions. Both the Central 
Difference Scheme and Ilin Scheme will be compared regarding their accuracy.

\subfile{../src/02_eps_to_zero}

\subfile{../src/03_n_acc}

\subfile{../src/04_eps_acc}

\subfile{../src/05_s_acc}

\newpage
\section{Conclusion}

To conclude, both the Central Difference Scheme and the Ilin Scheme approximate
problems \eqref{eq:bvp1}, \eqref{eq:bvp2}, and \eqref{eq:bvp3} well. When $n$ 
is increased, the accuracy of the approximation increases, too. Both CDS and IS
converge to the analytic function for large--enough $n$. For low $n$ CDS tends
to oscillate and thus IS is a better approximation for those cases. Even for
Robin problems which are difficult to approximate at the boundary layer high
$n$ lead to convergence for both schemes.

The parameter $\epsilon$ has a negative influence on the accuracy of the
numeric approximation when it becomes smaller. The shape of the function 
becomes more extreme and the finite difference methods become less accurate. IS 
can still approximate these cases well given the appropriate number of
nodes. CDS has more trouble because it oscillates, especially around the
boundary layer, making the approximation imprecise.

All in all I think that IS is the better method to approximate the solutions
this report covers. It does not suffer from inaccuracy caused by oscillations
if $n$ or $\epsilon$ is small. It converges slower than CDS and is
more computationally expensive because of the involved exponential functions,
but in my opinion that is worth it. The only drawback that I found in my
analysis is that for \textsc{Problem 3} with small $\epsilon$ IS is not
monotonous and did not produce a result. On the other hand, the result obtained
by CDS was 3.7 times off the mark which indicates that neither method can
approximate very small $\epsilon$ well.

\end{document}
