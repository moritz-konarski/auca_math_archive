\documentclass[../00_main.tex]{subfiles}

\begin{document}

\subsection{Behavior as $\epsilon$ Tends to Zero}

As $\epsilon$ tends to zero the shape of all three functions becomes more
pronounced. What I mean is that when $\epsilon$ is large ($\epsilon \ge 0.5$),
the functions have more smoothly changing slopes. When $\epsilon$ approaches 
zero, the functions have more abrupt changes in their slope. These
abrupt changes in slope take place at the boundary layers of the functions 
which are at $x=0$ for all three problems. \figref{fig:bvp1_e} shows this 
behavior for \eqref{eq:bvp1}. It shall also represent \eqref{eq:bvp2} because  
even though they are different functions their graphs are similar enough for 
this illustration. \figref{fig:bvp3_e} shows this behavior for \eqref{eq:bvp3}, 
showing that this problem also develops a clear boundary layer at 
$x=0$. For \eqref{eq:bvp1} and \eqref{eq:bvp2} we have $\phi_0=0.25$ and 
$\phi_1=0.75$ as the boundary values.
\begin{figure}[H]
\centering\begin{tikzpicture}
\begin{axis}[
    title={\textsc{Problem 1} for $\epsilon=0.5$},
    width=7cm, height=7cm,
    line width=1pt,
    grid = major,
    grid style = {dashed, gray!30},
    xmin = -0.1,
    xmax = 1.1,
    ymin = -0.1,
    ymax = 1.1,
    /pgfplots/xtick = {-0.2, 0.0,...,1.2},
    axis background/.style = {fill=white},
    ylabel = {$u(x)$},
    xlabel = {$x$},
    legend style={legend pos=north east},
    tick align = outside,]
    \addplot [red, mark=none] file {"../beh_e/BVP 1 (Dirichlet)_Central
    Difference_3_e-0.5_p1-0.25_p2-0.75_analytic.csv"};
\legend{\textit{analytic}}
\end{axis}%
\end{tikzpicture}\qquad
\begin{tikzpicture}
\begin{axis}[
    title={\textsc{Problem 1} for $\epsilon=0.0125$},
    width=7cm, height=7cm,
    line width=1pt,
    grid = major,
    grid style = {dashed, gray!30},
    xmin = -0.1,
    xmax = 1.1,
    ymin = -0.1,
    ymax = 1.1,
    /pgfplots/xtick = {-0.2, 0.0,...,1.2},
    axis background/.style = {fill=white},
    ylabel = {$u(x)$},
    xlabel = {$x$},
    legend style={legend pos=north east},
    tick align = outside,]
    \addplot [red, mark=none] file {"../beh_e/BVP 1 (Dirichlet)_Central
    Difference_3_e-0.0125_p1-0.25_p2-0.75_analytic.csv"};
\legend{\textit{analytic}}
\end{axis}%
\end{tikzpicture}
    \vspace{-20pt}
    \caption{Behavior of \eqref{eq:bvp1} for $\phi_0=0.25$, 
    $\phi_1=0.75$, different $\epsilon$}
\label{fig:bvp1_e}
\end{figure}

\begin{figure}[h]
\centering\begin{tikzpicture}
\begin{axis}[
    title={\textsc{Problem 3} for $\epsilon=0.5$},
    width=7cm, height=7cm,
    line width=1pt,
    grid = major,
    grid style = {dashed, gray!30},
    xmin = -0.1,
    xmax = 1.1,
    ymin = -0.1,
    ymax = 1.1,
    /pgfplots/xtick = {-0.2, 0.0,...,1.2},
    axis background/.style = {fill=white},
    ylabel = {$u(x)$},
    xlabel = {$x$},
    legend style={legend pos=north east},
    tick align = outside,]
    \addplot [red, mark=none] file {"../beh_e/Problem 3 (R)_Central
    Difference_3_e-0.5_p1-0.25_p2-0.75_analytic.csv"};
    \legend{\textit{analytic}}
\end{axis}%
\end{tikzpicture}\qquad
\begin{tikzpicture}
\begin{axis}[
    title={\textsc{Problem 3} for $\epsilon=0.0125$},
    width=7cm, height=7cm,
    line width=1pt,
    grid = major,
    grid style = {dashed, gray!30},
    xmin = -0.1,
    xmax = 1.1,
    ymin = -0.1,
    ymax = 1.1,
    /pgfplots/xtick = {-0.2, 0.0,...,1.2},
    axis background/.style = {fill=white},
    ylabel = {$u(x)$},
    xlabel = {$x$},
    legend style={legend pos=north east},
    tick align = outside,]
    \addplot [red, mark=none] file {"../beh_e/Problem 3 (R)_Central
    Difference_3_e-0.0125_p1-0.25_p2-0.75_analytic.csv"};
    \legend{\textit{analytic}}
    \end{axis}%
\end{tikzpicture}
    \vspace{-20pt}
    \caption{Behavior of \eqref{eq:bvp3} for different 
    $\epsilon$}
\label{fig:bvp3_e}
\end{figure}

\figref{fig:bvp1_e} and \figref{fig:bvp3_e} allow us to conclude that all three 
functions have a boundary layer at $x=0$ and that a smaller $\epsilon$ leads to 
a more pronounced boundary layer with a more drastic change in slope. 

\end{document}
