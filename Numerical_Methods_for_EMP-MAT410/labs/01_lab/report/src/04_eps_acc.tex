\documentclass[../00_main.tex]{subfiles}

\begin{document}

\subsection{Effect of $\epsilon$ on Accuracy}

Epsilon has a negative effect on the accuracy of approximations when
it gets smaller. This is caused by the more extreme shape of the functions 
around the boundary layer that makes is harder to successfully approximate 
them. Especially the CDS has difficulties here as it starts to oscillate around 
the drastic change in slope near the boundary layer. These oscillations then 
continue for the length of the function, making the whole approximation 
imprecise. IS can successfully avoid these oscillations as the graphs below 
illustrate. The exponential functions involved in the hyperbolic cotangent in 
$R_i$ of IS are the reason for this.

\figref{fig:err_cds_eps} shows that when $\epsilon \rightarrow 0$ the error
of CDS increases dramatically ($n=33$ here). For \textsc{Problem 1} and 
\textsc{Problem 2} the maximum error for $\epsilon=0.0001$ is 164.2\% and
41.4\% respectively. For \textsc{Problem 3} the maximum error for this value
of $\epsilon$ is 367.7\% and even for $\epsilon=0.001$ the error is 199\%.
Neither of these values fit on the graph.\\
\begin{figure}[h]
\centering\begin{tikzpicture}
\begin{axis}[
    title={CDS Error Dependent on $\epsilon$},
    width=15cm, height=10cm,
    line width=1pt,
    grid = major,
    grid style = {dashed, gray!30},
    xmin = -0.1,
    xmax = 1.1,
    ymin = -5,
    ymax = 170,
    /pgfplots/xtick = {-0.1, 0,...,1.1},
    axis background/.style = {fill=white},
    ylabel = {$Error\text{ }in\text{ }\%$},
    xlabel = {$\epsilon$},
    legend style={legend pos=north east},
    tick align = outside,]

    \addplot [red, mark=square] file {"../acc_e/BVP 1 (Dirichlet)_Central
    Difference_33_eps_p1-0.25_p2-0.75_errors.dat"};
    \addplot [blue, mark=square] file {"../acc_e/BVP 2 (Robin)_Central
    Difference_33_eps_p1-0.25_p2-0.75_errors.dat"};
    \addplot [green, mark=square] file {"../acc_e/Problem 3 (R)_Central
    Difference_33_eps_p1-0.25_p2-0.75_errors.dat"};

    \legend{\textsc{Problem 1},\textsc{Problem 2},\textsc{Problem 3}}
    \end{axis}%
\end{tikzpicture}
    \vspace{-20pt}
    \caption{Accuracy of CDS dependent on $\epsilon$.
    $\phi_0=0.25$, $\phi_1=0.75$, $n=33$}
\label{fig:err_cds_eps}
\end{figure}
The large error in \textsc{Problem 3} can be explained by the aforementioned
oscillations that are caused by the pronounced boundary layer in that problem. 
To illustrate this, \figref{fig:bvp3_acc_e} shows the analytic and numerical 
solutions for $n=33$ and $\epsilon=0.1$ and $\epsilon=0.001$ respectively. The 
graph on the right shows the oscillations that explain the high error.
\begin{figure}[H]
\centering\begin{tikzpicture}
\begin{axis}[
    title={\textsc{Problem 3} for $\epsilon=0.1$},
    width=7cm, height=7cm,
    line width=1pt,
    grid = major,
    grid style = {dashed, gray!30},
    xmin = -0.1,
    xmax = 1.1,
    ymin = -0.1,
    ymax = 3.1,
    /pgfplots/xtick = {-0.2, 0.0,...,1.2},
    axis background/.style = {fill=white},
    ylabel = {$u(x)$},
    xlabel = {$x$},
    legend style={legend pos=north east},
    tick align = outside,]
    \addplot [blue, mark=square] file {"../acc_e/Problem 3 (R)_Central 
    Difference_33_e-0.1_p1-0.25_p2-0.75_numerical.dat"};
    \addplot [red, mark=none] file {"../acc_e/Problem 3 (R)_Central 
    Difference_33_e-0.1_p1-0.25_p2-0.75_analytic.dat"};
    \legend{\textit{numeric},\textit{analytic}}
\end{axis}%
\end{tikzpicture}\qquad
\begin{tikzpicture}
\begin{axis}[
    title={\textsc{Problem 3} for $\epsilon=0.001$},
    width=7cm, height=7cm,
    line width=1pt,
    grid = major,
    grid style = {dashed, gray!30},
    xmin = -0.1,
    xmax = 1.1,
    ymin = -0.1,
    ymax = 3.1,
    /pgfplots/xtick = {-0.2, 0.0,...,1.2},
    axis background/.style = {fill=white},
    ylabel = {$u(x)$},
    xlabel = {$x$},
    legend style={legend pos=north east},
    tick align = outside,]
    \addplot [blue, mark=square] file {"../acc_e/Problem 3 (R)_Central 
    Difference_33_e-0.001_p1-0.25_p2-0.75_numerical.dat"};
    \addplot [red, mark=none] file {"../acc_e/Problem 3 (R)_Central 
    Difference_33_e-0.001_p1-0.25_p2-0.75_analytic.dat"};
    \legend{\textit{numeric},\textit{analytic}}
\end{axis}%
\end{tikzpicture}
    \vspace{-20pt}
\caption{Behavior of \eqref{eq:bvp3} with CDS for $n=33$}
\label{fig:bvp3_acc_e}
\end{figure}

While CDS is fairly accurate for most values of $\epsilon$, when it becomes too
small, the numerical solution begins to oscillate and becomes very inaccurate. 
IS does not face that problem as it's $R_i$ is not prone to oscillations.
\figref{fig:err_is_eps} shows the errors for $\epsilon \rightarrow 0$ for IS. 
We see that the error of IS also increases, but much less than for CDS ($n=33$ 
also here). For \textsc{Problem 1} and \textsc{Problem 2} the maximum error for 
$\epsilon=0.0001$ is 8.5\% and 3.8\% respectively. For \textsc{Problem 3} 
the maximum error for this value can't be found with my program. It returns
\texttt{NaN} (not a number). I think the reason for this is that 
\textsc{Problem 3} is not monotone for any values of $\epsilon$ that I tested.
This causes the numeric solution to break down for small $\epsilon$.
Furthermore, the conditions for stability are also not met. For 
$\epsilon=0.001$ the error is 39.4\% for this problem.

The explanation for the great increase in accuracy compared to CDS is that IS 
does not start to oscillate for small values of $\epsilon$ and thus keeps its 
accuracy. \figref{fig:err_is_eps} shows the much smaller errors for small 
$\epsilon$ of IS compared to CDS (note that the axes are the same as in 
\figref{fig:err_cds_eps}). \\\figref{fig:bvp3_acc_e_is} shows the same problem 
and parameters as \figref{fig:bvp3_acc_e} and demonstrates that IS does not 
oscillate in a situation where CDS did.
\begin{figure}[h]
\centering\begin{tikzpicture}
\begin{axis}[
    title={IS Error Dependent on $\epsilon$},
    width=15cm, height=10cm,
    line width=1pt,
    grid = major,
    grid style = {dashed, gray!30},
    xmin = -0.1,
    xmax = 1.1,
    ymin = -5,
    ymax = 170,
    /pgfplots/xtick = {-0.1, 0,...,1.1},
    axis background/.style = {fill=white},
    ylabel = {$Error\text{ }in\text{ }\%$},
    xlabel = {$\epsilon$},
    legend style={legend pos=north east},
    tick align = outside,]

    \addplot [red, mark=square] file {"../acc_e/BVP 1 (Dirichlet)_Ilin
    Scheme_33_eps_p1-0.25_p2-0.75_errors.dat"};
    \addplot [blue, mark=square] file {"../acc_e/BVP 2 (Robin)_Ilin
    Scheme_33_eps_p1-0.25_p2-0.75_errors.dat"};
    \addplot [green, mark=square] file {"../acc_e/Problem 3 (R)_Ilin
    Scheme_33_eps_p1-0.25_p2-0.75_errors.dat"};

    \legend{\textsc{Problem 1},\textsc{Problem 2},\textsc{Problem 3}}
    \end{axis}%
\end{tikzpicture}
    \vspace{-20pt}
    \caption{Accuracy of IS dependent on $\epsilon$. $\phi_0=0.25$,
    $\phi_1=0.75$, $n=33$}
\label{fig:err_is_eps}
\end{figure}

\begin{figure}[H]
\centering\begin{tikzpicture}
\begin{axis}[
    title={\textsc{Problem 3} for $\epsilon=0.1$},
    width=7cm, height=7cm,
    line width=1pt,
    grid = major,
    grid style = {dashed, gray!30},
    xmin = -0.1,
    xmax = 1.1,
    ymin = -0.3,
    ymax = 1.1,
    /pgfplots/xtick = {-0.2, 0.0,...,1.2},
    axis background/.style = {fill=white},
    ylabel = {$u(x)$},
    xlabel = {$x$},
    legend style={legend pos=south east},
    tick align = outside,]
    \addplot [blue, mark=square] file {"../acc_e/Problem 3 (R)_Ilin 
    Scheme_33_e-0.1_p1-0.25_p2-0.75_numerical.dat"};
    \addplot [red, mark=none] file {"../acc_e/Problem 3 (R)_Ilin 
    Scheme_33_e-0.1_p1-0.25_p2-0.75_analytic.dat"};
    \legend{\textit{numeric},\textit{analytic}}
\end{axis}%
\end{tikzpicture}\quad
\begin{tikzpicture}
\begin{axis}[
    title={\textsc{Problem 3} for $\epsilon=0.001$},
    width=7cm, height=7cm,
    line width=1pt,
    grid = major,
    grid style = {dashed, gray!30},
    xmin = -0.1,
    xmax = 1.1,
    ymin = -0.3,
    ymax = 1.1,
    /pgfplots/xtick = {-0.2, 0.0,...,1.2},
    axis background/.style = {fill=white},
    ylabel = {$u(x)$},
    xlabel = {$x$},
    legend style={legend pos=south east},
    tick align = outside,]
    \addplot [blue, mark=square] file {"../acc_e/Problem 3 (R)_Ilin 
    Scheme_33_e-0.001_p1-0.25_p2-0.75_numerical.dat"};
    \addplot [red, mark=none] file {"../acc_e/Problem 3 (R)_Ilin 
    Scheme_33_e-0.001_p1-0.25_p2-0.75_analytic.dat"};
    \legend{\textit{numeric},\textit{analytic}}
\end{axis}%
\end{tikzpicture}
    \vspace{-20pt}
\caption{Behavior of \eqref{eq:bvp3} with IS for $n=33$}
\label{fig:bvp3_acc_e_is}
\end{figure}

We can see that as $\epsilon \rightarrow 0$ approximating the analytical
functions becomes more and more difficult. The parameter $\epsilon$ has 
negative influence on the convergence of the numerical solution to the 
analytical solution when it decreases. CDS begins to strongly oscillate
around the boundary layer when $\epsilon$ is small and the approximation
becomes pretty useless. IS does not have oscillations, but for \textsc{Problem
3} the numerical approximation is neither monotone nor stable which means that 
a solution might not exist for certain parameters. This is what happened when 
$\epsilon=0.0001$.

\end{document}
