\documentclass[../00_main.tex]{subfiles}

\begin{document} 

\subsection{Accuracy as $n$ Increases}

When the number of nodes $n$ increases, the accuracy of the
approximation increases, too. In other words, the numerical solution converges
to the analytic solution as $n$ increases. The error of the approximation is 
calculated using the formula
\begin{equation}\nonumber
    \text{Err} \equiv \frac{\underset{1 \leq i \leq n}{\max}|u(x_i)
        - u^h_i|}{\underset{1 \leq i \leq n}{\max}|u(x_i)|} \cdot 100\%.
\end{equation}
A small error indicates that the numeric approximations are converging to the 
analytic solutions. In \figref{fig:err_cds} and \figref{fig:err_is} the error 
of CDS and IS approximations dependent on $n$ is shown. \figref{fig:err_cds} 
illustrates that CDS is not very accurate for low $n$, with \textsc{Problem 3} 
even having an error of $402.26\%$ for $n=3$. But as $n$ increases, the error 
decreases quickly. For the first two problems it is below 10\% at only 17 
nodes. For \textsc{Problem 3} is takes 64 nodes to reach the same accuracy 
which suggests that it is harder to approximate this problem than the first two 
problems.
\begin{figure}[h]
\centering\begin{tikzpicture}
\begin{axis}[
    title={CDS Error Dependent on $n$},
    width=15cm, height=10cm,
    line width=1pt,
    grid = major,
    grid style = {dashed, gray!30},
    xmin = -5,
    xmax = 520,
    ymin = -5,
    ymax = 120,
    /pgfplots/xtick = {0, 50,...,520},
    axis background/.style = {fill=white},
    ylabel = {$Error\text{ }in\text{ }\%$},
    xlabel = {$n$},
    legend style={legend pos=north east},
    tick align = outside,]

    \addplot [red, mark=square] file {"../acc_n/BVP 1 (Dirichlet)_Central
    Difference_steps_e-0.05_p1-0.25_p2-0.75_errors.dat"};
    \addplot [blue, mark=square] file {"../acc_n/BVP 2 (Robin)_Central
    Difference_steps_e-0.05_p1-0.25_p2-0.75_errors.dat"};
    \addplot [green, mark=square] file {"../acc_n/Problem 3 (R)_Central 
    Difference_steps_e-0.05_p1-0.25_p2-0.75_errors.dat"};

    \legend{\textsc{Problem 1},\textsc{Problem 2},\textsc{Problem 3}}
    \end{axis}%
\end{tikzpicture}
    \vspace{-20pt}
    \caption{Accuracy of CDS dependent on $n$. 
    $\phi_0=0.25$, $\phi_1=0.75$, $\epsilon=0.05$}
\label{fig:err_cds}
\end{figure}

\figref{fig:err_is} shows that IS is more accurate for low $n$ than CDS, where
the largest error is again \textsc{Problem 3} with $n=5$ at $38\%$. As $n$ 
increases, this error decreases quickly, too. Again the first two problems 
converge faster than the third problem. 

\figref{fig:err_cds} and \figref{fig:err_is} illustrate that \textsc{Problem 3} 
is harder to approximate than \textsc{Problem 1} or \textsc{Problem 2}. This is 
most likely caused by \textsc{Problem 3}'s more pronounced boundary layer which 
is harder to approximate. Furthermore we see that if $n$ is large enough, the 
numeric solution converges to the analytic solution. As an example of this see 
\figref{fig:bvp2_acc_n} which shows that for large $n$, the numeric solution 
using IS converges to the analytical one, even though the problem's Robin 
boundary condition is not handled well for low $n$.

\begin{figure}[h]
\centering\begin{tikzpicture}
\begin{axis}[
    title={IS Error Dependent on $n$},
    width=15cm, height=10cm,
    line width=1pt,
    grid = major,
    grid style = {dashed, gray!30},
    xmin = -5,
    xmax = 520,
    ymin = -5,
    ymax = 120,
    /pgfplots/xtick = {0, 50,...,520},
    axis background/.style = {fill=white},
    ylabel = {$Error\text{ }in\text{ }\%$},
    xlabel = {$n$},
    legend style={legend pos=north east},
    tick align = outside,]

    \addplot [red, mark=square] file {"../acc_n/BVP 1 (Dirichlet)_Ilin 
    Scheme_steps_e-0.05_p1-0.25_p2-0.75_errors.dat"};
    \addplot [blue, mark=square] file {"../acc_n/BVP 2 (Robin)_Ilin 
    Scheme_steps_e-0.05_p1-0.25_p2-0.75_errors.dat"};
    \addplot [green, mark=square] file {"../acc_n/Problem 3 (R)_Ilin 
    Scheme_steps_e-0.05_p1-0.25_p2-0.75_errors.dat"};

    \legend{\textsc{Problem 1},\textsc{Problem 2},\textsc{Problem 3}}
    \end{axis}%
\end{tikzpicture}
    \vspace{-20pt}
    \caption{Accuracy of IS dependent on $n$. $\phi_0=0.25$, 
    $\phi_1=0.75$, $\epsilon=0.05$}
\label{fig:err_is}
\end{figure}

\begin{figure}[H]
\centering\begin{tikzpicture}
\begin{axis}[
    title={\textsc{Problem 2} for $n=3$},
    width=7cm, height=7cm,
    line width=1pt,
    grid = major,
    grid style = {dashed, gray!30},
    xmin = -0.1,
    xmax = 1.1,
    ymin = -0.1,
    ymax = 1.1,
    /pgfplots/xtick = {-0.2, 0.0,...,1.2},
    axis background/.style = {fill=white},
    ylabel = {$u(x)$},
    xlabel = {$x$},
    legend style={legend pos=north east},
    tick align = outside,]
    \addplot [blue, mark=square] file {"../acc_n/BVP 2 (Robin)_Ilin
    Scheme_3_e-0.05_p1-0.25_p2-0.75_numerical.dat"};
    \addplot [red, mark=none] file {"../acc_n/BVP 2 (Robin)_Ilin 
    Scheme_3_e-0.05_p1-0.25_p2-0.75_analytic.dat"};
    \legend{\textit{numeric},\textit{analytic}}
\end{axis}%
\end{tikzpicture}\qquad
\begin{tikzpicture}
\begin{axis}[
    title={\textsc{Problem 2} for $n=257$},
    width=7cm, height=7cm,
    line width=1pt,
    grid = major,
    grid style = {dashed, gray!30},
    xmin = -0.1,
    xmax = 1.1,
    ymin = -0.1,
    ymax = 1.1,
    /pgfplots/xtick = {-0.2, 0.0,...,1.2},
    axis background/.style = {fill=white},
    ylabel = {$u(x)$},
    xlabel = {$x$},
    legend style={legend pos=north east},
    tick align = outside,]
    \addplot [blue, mark=square] file {"../acc_n/BVP 2 (Robin)_Ilin
    Scheme_257_e-0.05_p1-0.25_p2-0.75_numerical.dat"};
    \addplot [red, mark=none] file {"../acc_n/BVP 2 (Robin)_Ilin 
    Scheme_257_e-0.05_p1-0.25_p2-0.75_analytic.dat"};
    \legend{\textit{numeric},\textit{analytic}}
\end{axis}%
\end{tikzpicture}
    \vspace{-20pt}
\caption{Convergence of \eqref{eq:bvp2} with IS for $\phi_0=0.25$,
   $\phi_1=0.75$, $\epsilon=0.05$}
\label{fig:bvp2_acc_n}
\end{figure}

Considering the effect of $n$ on the accuracy of the numerical approximation we 
can say that an increase in $n$ leads to a decrease in the error and a better 
convergence of the numerical solution to analytical solution. 

\end{document}
