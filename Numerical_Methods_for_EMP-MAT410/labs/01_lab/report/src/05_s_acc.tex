\documentclass[../00_main.tex]{article}

\begin{document}

\subsection{Accuracy of Difference Schemes}

The two schemes discussed in this report both have advantages and
disadvantages which will be discussed using \textsc{Problem 3} as the example
because it was shown to be the most difficult problem to approximate. 

When $\epsilon$ is decreased our previous analysis has shown that it becomes
harder to approximate the functions because of the shape of the function near
the boundary layer. \tabref{tab:err_eps} shows the error in \% of the
numerical solution compared to the analytical solution for \textsc{Problem 3} 
and both schemes. The last column of the table is the ratio of the error of CDS 
to the error of IS. This enables us to more easily compare the schemes. 
\begin{table}[h]
\centering
\begin{tabular}{| r | r | r | r |}\hline
    $\epsilon$  & CDS Error in \%   & IS Error in \% & $\text{Err}_{\text{CDS}}
    /\text{Err}_{\text{IS}}$   \\\hline\hline
    1       &   0.07529  &   0.10654  &  0.70668    \\\hline
    0.5     &   0.20207  &   0.22170  &  0.91145    \\\hline
    0.25    &   0.61485  &   0.68392  &  0.89900    \\\hline
    0.1     &   3.48381  &   3.79717  &  0.91747    \\\hline
    0.05    &  13.43621  &  11.39853  &  1.17876    \\\hline
    0.01    &  66.91237  &  33.49386  &  1.99775    \\\hline
    0.005   &  87.29208  &  36.84592  &  2.36911    \\\hline
    0.001   & 199.07709  &  39.39860  &  5.05289    \\\hline
    0.0001  & 367.72195  &  \texttt{NaN} &   \texttt{NaN} \\\hline
\end{tabular}
    \vspace{15pt}
    \caption{Errors in \% for \eqref{eq:bvp3} for CDS and IS, various
    $\epsilon$, $n=33$}
    \label{tab:err_eps}
\end{table}
\tabref{tab:err_eps} shows that for large $\epsilon$ CDS is the better
approximation but as $\epsilon$ decreases their accuracies come closer. This
trend continues and IS becomes more accurate than CDS. This is caused by the
oscillations that CDS experiences for small $\epsilon$ making the
approximation inaccurate. As IS does not experience oscillations, its error
increases much slower. For \textsc{Problem 3} we see that IS can
have limits, as it is neither monotone nor stable for this exercise and these 
parameters. This
means that the solution is not stable and thus when $\epsilon$ gets too small
my program cannot find a solution. 

The other important parameter in this problem is $n$, the number of nodes. In
\tabref{tab:err_n} we see the error for the CDS and IS approximations of
\textsc{Problem 3} and again their ratio. 
\begin{table}[h]
\centering
\begin{tabular}{| r | r | r | r |}\hline
    $n$  & CDS Error in \%   & IS Error in \% & $\text{Err}_{\text{CDS}}
    /\text{Err}_{\text{IS}}$   \\\hline\hline
    3      & 402.26497 & 31.32132 & 12.84316    \\\hline
    5      & 225.19784 & 38.23313 & 5.89012     \\\hline
    17     &  36.95902 & 24.19590 & 1.52749     \\\hline
    33     &  13.43621 & 11.39853 & 1.17876     \\\hline
    65     &   3.24558 &  3.74921 & 0.86567     \\\hline
    129    &   0.74625 &  1.00782 & 0.74045     \\\hline
    257    &   0.18301 &  0.26429 & 0.69245     \\\hline
    513    &   0.04564 &  0.06782 & 0.67295     \\\hline
    1000   &   0.01198 &  0.01805 & 0.66371     \\\hline
    10000  &   0.00012 &  0.00018 & 0.66666     \\\hline
\end{tabular}
    \vspace{15pt}
    \caption{Errors in \% for \eqref{eq:bvp3} for CDS and IS, various
    $n$, $\epsilon=0.05$}
    \label{tab:err_n}
\end{table}
\tabref{tab:err_n} demonstrates that for small $n$ IS is more accurate because
CDS tends to oscillate more for low $n$. Then, as $n$ increases, CDS becomes 
more accurate. As $n$ increases further the two methods begin to converge to an 
error of 0\% and the difference between them continues to increase but the rate
of change slows down. This means that when $n$ increases, CDS converges a little
bit faster than IS, but they both converge to the analytical solution. 

\end{document}
