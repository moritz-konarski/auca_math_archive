% test to copy pages 545-546 from Kreyzig's text book to practice markdown

% TODO: - align equations correctly
%       - add pictures
%       - number pages according to book
%       - add running head to pages
%       - fix position of page numbers
%       - fix equivalence sign at (8)
%       - add color boxes to appropriate equations

\documentclass[twocolumn, 12pt, leqno, oneside]{amsart}

\usepackage[ansiapaper]{geometry}
\usepackage{amsfonts}
\usepackage{amsmath}
\usepackage{anysize}
\usepackage{xcolor}

\renewcommand{\baselinestretch}{1.3}

\marginsize{2cm}{2cm}{2cm}{2cm}

\setlength{\columnsep}{1cm}

\begin{document}

\section{Solution by Separating Variables.\\
Use of Fourier Series}

We continue our work from Sec. 12.2, where we modeled a vibrating string and 
obtained the one-dimensional wave equation. We now have to complete the model 
by adding additional conditions and then solving the resulting model.

The model of a vibrating elastic string (a violin string, for instance)
consists of the \textbf{one-dimensional wave equation}
\begin{equation}
    \frac{\partial^2{u}}{\partial{t^2}} = c^2\frac{\partial^2{u} \\
        }{\partial{x^2}} \qquad \qquad c^2 = \frac{T}{\rho}
\end{equation}
for the unknown deflection $u(x,t)$ of the string, a PDE that we have just 
obtained, and some \textbf{\emph{additional conditions}}, which we shall now 
derive.

Since the string is fastened at the ends $x=0$ and $x=L$ (see Sec. 12.2), we 
have the two \textbf{boundary conditions}
\begin{equation}
  \begin{split}
       &\text{(a)} \quad u(0,t) = 0, \\
       &\text{(b)} \quad u(L,t) = 0, \\
       &\text{for all } t \geq 0.
  \end{split}
\end{equation}
Furthermore, the form of the motion of the string will depend on its
\emph{initial deflection} (deflection at time $t=0$), call it $f(x)$ and on 
its \emph{initial velocity} (velocity at $t=0$), call it $g(x)$.
We thus have the two \textbf{initial conditions}
\begin{equation}
  \begin{split}
        &\text{(a)} \quad u(x,0) = f(x), \\
        &\text{(b)} \quad u_t(x,0) = g(x) \\
        & (0 \leq x \leq L)
  \end{split}
\end{equation}
where $u_t=\partial u/\partial t$. We now have to find a solution of the PDE 
(1) satisfying the conditions (2) and (3). This will be the solution of our 
problem. We shall do this in three steps, as follows.

\emph{\textbf{Step 1.}} By the “\textbf{method of separating variables}” or
\emph{product method}, setting $u(x,t)=F(x)G(t)$, we obtain from (1) two 
ODEs, one for $F(x)$ and the other one for $G(t)$.

\emph{\textbf{Step 2.}} We determine solutions of these ODEs that satisfy 
the boundary conditions (2).

\emph{\textbf{Step 3.}} Finally, using \textbf{Fourier series}, we compose 
the solutions found in Step 2 to obtain a solution of (1) satisfying both 
(2) and (3), that is, the solution of our model of the vibrating string.

\subsection{Step 1. Two ODEs from the Wave Equation (1)}

In the \textbf{method of separating variables}, or \emph{product method}, we 
determine solutions of the wave equation (1) of the form
\begin{equation}
    u(x, t) = F(x)G(x)
\end{equation}
which are a product of two functions, each depending on only one of the 
variables $x$ and $t$. This is a powerful general method that has various 
applications in engineering mathematics, as we shall see in this chapter. 
Differentiating (4), we obtain
\begin{equation}
  \nonumber
    \frac{ \partial^2{u} }{ \partial{t^2} } = F\ddot{G} \qquad \text{and} \\
        \qquad \frac{ \partial^2{u} }{ \partial{x^2} } = F''G
\end{equation}
where dots denote derivatives with respect to $t$ and primes derivatives with
respect to $x$. By inserting this into the wave equation (1) we have
\begin{equation}
  \nonumber
    F\ddot{G} = c^2F''G.
\end{equation}
Dividing by $c^2FG$ and simplifying gives
\begin{equation}
  \nonumber
    \frac{\ddot{G}}{c^2G} = \frac{F''}{F}.
\end{equation}
The variables are now separated, the left side depending only on $t$ and the 
right side only on $x$. Hence both sides must be constant because, if they 
were variable, then changing $t$ or $x$ would affect only one side, leaving 
the other unaltered. Thus, say,
\begin{equation}
  \nonumber
    \frac{\ddot{G}}{c^2G} = \frac{F''}{F} = k.
\end{equation}
Multiplying by the denominators gives immediately two \emph{\textbf{ordinary}} 
DEs
\begin{equation}
    F'' - kF = 0
\end{equation}
and
\begin{equation}
    \ddot{G} - c^2kG = 0.
\end{equation}
Here, the \textbf{separation constant} \emph{k} is still arbitrary.

\subsection{Step 2. Satisfying the Boundary Conditions (2)}

We now determine solutions $F$ and $G$ of (5) and (6) so that $u=FG$ satisfies 
the boundary conditions (2), that is,
\begin{equation}
  \begin{split}
      & u(0,t)=F(0)G(t) =0, \\
      & u(L,r)=F(L)G(t) =0 \\
      & \text{for all }t.
  \end{split}
\end{equation}
We first solve (5). If $G=0$, then $u=FG\equiv0$, which is of no interest. 
Hence $G\neq0$ and then by (7),
\begin{equation}
    \text{(a)} \quad F(0)=0, \qquad \text{(b)} \quad F(L)=0.
\end{equation}
We show that $k$ must be negative. For $k=0$ the general solution of (5) is
$F=ax+b$, and from (8) we obtain $a=b=0$, so that $F\equiv0$ and 
$u=FG\equiv0$, which is of no interest. For positive $k=\mu^2$ a general 
solution of (5) is
\begin{equation}
  \nonumber
    F=Ae^{\mu x} = Be^{-\mu x}
\end{equation}
and from (8) we obtain $F\equiv0$ as before (verify!). Hence we are left with 
the possibility of choosing $k$ negative, say, $k=-p^2$. Then (5) becomes 
$F''+p^2F=0$ and has as a general solution
\begin{equation}
  \nonumber
    F(x)=A\cos{px} + B\sin{px}.
\end{equation}
From this and (8) we have
\begin{equation}
  \nonumber
  \begin{split}
      & F(0)=A=0 \qquad \text{and then} \\ 
      & F(L)=B\sin{pL}=0
  \end{split}
\end{equation}
We must take $B\neq0$ since otherwise $F\equiv0$. Hence $\sin{pL}=0$. Thus
\begin{equation}
  \begin{split}
      pL&=n\pi, \qquad \text{so that} \\
      p&=\frac{n\pi}{L} \qquad \text{(\emph{n} integer)}.
  \end{split}
\end{equation}
Setting $B=1$, we thus obtain infinitely many solutions $F(x)=F_n(x)$, where
\begin{equation}
    F_n(x)=\sin{ \frac{n\pi}{L} x} \quad \quad (n=1,2,\dotsb).
\end{equation}
These solutions satisfy (8). [For negative integer $n$ we obtain essentially 
the same solutions, except for a minus sign, because 
$\sin{(-\alpha)}=-\sin{\alpha}$.]

We now solve (6) with \newline $k=-p^2=-(n\pi/L)^2$ resulting from (9), that is
\begin{equation}
  \tag{11*}
    \ddot{G}+\lambda_n^2G=0 \quad \text{where} \quad \lambda_n=cp= \\
        \frac{cn\pi}{L}.
\end{equation}
A general solution is
\begin{equation}
  \nonumber
    G_n(t)=B_n\cos{\lambda_n t} + B_n^\star\sin{\lambda_n t}.
\end{equation}
Hence solutions of (1) satisfying (2) are 
$u_n(x,t)=F_n(x)G_n(t)=G_n(t)F_n(x)$, written out
\begin{equation}
  \begin{split}
      u_n(x,t)=&(B_n\cos{\lambda_n t} + B_n^\star\sin{\lambda_n t}) \\
        &\sin{\frac{n\pi}{L} x} \quad \quad (n=1,2,\dotsb).
  \end{split}
\end{equation}
These functions are called the \textbf{eigenfunctions}, or 
\emph{characteristic functions}, and the values $\lambda_n=cn\pi/L$ are called 
the \textbf{eigenvalues}, or \emph{characteristic values}, of the vibrating 
string. The set $\{\lambda_1,\lambda_n,\dotsb\}$ is called the 
\textbf{spectrum}.

\subsubsection{Discussion of Eigenfunctions}

We see that each $u_n$ represents a harmonic motion having the 
\textbf{frequency} $\lambda_n/2\pi=cn/2L$ cycles per unit time. This motion is 
called the $n$th \textbf{normal mode} of the string. The first normal mode is 
known as the \emph{fundamental mode} $(n=1)$, and the others are known as 
\emph{overtones}; musically they give the octave, octave plus fifth, etc. 
Since in (11)
\begin{equation}
  \nonumber
  \begin{split}
      &\sin{\frac{n\pi x}{L}}=0 \qquad \text{at} \\
      & x=\frac{L}{n}, \frac{2L}{n},\dotsb,\frac{n-1}{n}L,
  \end{split}
\end{equation}
the $n$th normal mode has $n-1$ \textbf{nodes}, that is, points of the string 
that do not move (in addition to the fixed endpoints); see Fig. 287.

Figure 288 shows the second normal mode for various values of $t$. At any 
instant the string has the form of a sine wave. When the left part of the 
string is moving down, the other half is moving up, and conversely. For the 
other modes the situation is similar.

\textbf{Tuning} is done by changing the tension $T$. Our formula for the 
frequency $\lambda_n/2\pi=cn/2L$ of $u_n$  with $c=\sqrt{T/p}$ [see (3), Sec. 
12.2] confirms that effect because it shows that the frequency is proportional 
to the tension. $T$ cannot be increased indefinitely, but can you see what to 
do to get a string with a high fundamental mode? (Think of both $L$ and 
$\rho$.) Why is a violin smaller than a double-bass?
\end{document}
