\documentclass[a4paper,reqno,12pt]{article}

\usepackage[margin=1in]{geometry}
\usepackage[utf8]{inputenc}
\usepackage[english]{babel}
\usepackage{amsfonts}
\usepackage{amsmath}
\usepackage{amssymb}
\usepackage[allcolors=black,final]{hyperref}
\usepackage{graphicx}
\graphicspath{{../graphics/}}

\title{Topic Proposal for Report No. 2:\\Predator-Prey Models}
\author{Moritz M. Konarski}
\date{\today}

\begin{document}

\maketitle

\section{Topic}

For my second report of the semester I want to write about predator-prey
models. In the 1920s Volterra and Lotka independently derived the same system
of differential equations. Volterra sought to describe populations of two
interacting species (predator and prey fish), while Lotka tried to describe
oscillating chemical reactions \cite{chauvet}. In both cases it is simpler to
describe how populations of fish or concentrations of chemicals change than it
is to describe then directly and thus differential equations got involved. The 
Lotka-Volterra model was named in their honor:
\begin{equation}\nonumber
    \left\{\begin{aligned}
        &\frac{dx}{dt} = ax - bxy \\
        &\frac{dy}{dt} = -cy + dxy.
    \end{aligned}\right.
\end{equation}
Following \cite{chauvet}, $x(t)$ is the prey population and $y(t)$ is the
predator population. The other variables (all $>0$) are defined as:
\begin{itemize}
    \item $a$ -- natural growth rare of prey with no predators,
    \item $b$ -- effect of predators on prey (prey being eaten),
    \item $c$ -- natural death rate of predator with no prey,
    \item $d$ -- effect of prey on predator population (predators multiplying
        if they have food).
\end{itemize}

The Lotka-Volterra model will probably be the focus of my report, but other
predator-prey models like the Kermack-McKendrick model (accounts for herd
immunity and applies to epidemics) and the Jacob-Monod Model (accounts for
concentrations of e.g. bacteria and not number of animals) \cite{hoppensteadt} 
can be covered. A three-species food chain as discussed in \cite{chauvet} is
also interesting. If the unexpected nature of predator-prey models should be 
the focus of my report these equations would be a good thing to focus on.

\section{Elements of the Report}

\begin{itemize}
    \item Introduction: brief overview, introduction to topic, 
    \item History: how were the equations developed, by whom
    \item Finding solutions: with examples
        \begin{itemize}
            \item Analytic approaches
            \item Numerical approaches
        \end{itemize}
    \item Examples of one or more models computed by me
    \item Visualizations
        \begin{itemize}
            \item Graphs of solutions 
            \item Phase portraits 
        \end{itemize}
    \item Implications and Applications / Conclusion: what is the importance of 
        these equations, where are they applied (specifically)
\end{itemize}

\nocite{*}
\bibliography{../bib_02}
\bibliographystyle{abbrv}

\end{document}
