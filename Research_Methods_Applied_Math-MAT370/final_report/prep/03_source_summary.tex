\documentclass[a4paper,reqno,11pt]{article}

\usepackage[margin=1in]{geometry}
\usepackage[utf8]{inputenc}
\usepackage[english]{babel}
\usepackage{amsfonts}
\usepackage{amsmath}
\usepackage{amssymb}
\usepackage{amsrefs}
\usepackage[allcolors=black,final]{hyperref}
\usepackage{graphicx}
\graphicspath{{../graphics/}}

\title{Report No. 2: Predator-Prey Models\\Source Summary}
\author{Moritz M. Konarski}
\date{\today}

\begin{document}

\maketitle
\tableofcontents

\section{Chauvet: A Lotka-Volterra three-species food chain}

\subsection{Introduction}

\begin{itemize}
    \item Volterra described interactions between two competing species
    \item useful models to describe population variance over time
    \item standard equation: $a$ -- growth rate of prey w/o predators, $b$ --
        effects of predation on prey, $c$ -- death rate of predators w/o prey,
        $d$ -- increase of predators w/ prey; $a,b,c,d > 0$
        \begin{equation}\nonumber
            \left\{
            \begin{aligned}
                &\frac{dx}{dt} = ax - bxy \quad \text{(Prey)}\\
                &\frac{dy}{dt} = -cy +dxy \quad \text{(Predators)}
            \end{aligned}
            \right.
        \end{equation}
    \item divide the second equation by the first to get a linear ODE
    \item $C = a \ln y - by + c \ln x - dx$, max value of $C$ happens at
        $(a/d,a/b)$
    \item the plot moves counterclockwise because the predators lag behind the
        prey, which creates this movement -- phase-shifted behavior
    \item have common period, seen in historical records
    \item this paper defines a three-species non-logistic system of L-V
        equations
    \item normally logistic equations are used because the prey would otherwise
        grow unboundedly if there are no predators
    \item direct generalization of the standard equations
    \item 3 way system is quite complicated, good teaching tool because it has
        degeneracies -- trapping regions and other complicated shit
\end{itemize}

\subsection{The model}

\begin{itemize}
    \item 3 species food chain, $x$ is preyed on by $y$ which is preyed on by
        $z$ -- mouse, snake, owl; vegetation, hare, lynx; worm, robin, falcon
    \item model is:
        \begin{equation}\nonumber
            \left\{\begin{aligned}
                \frac{dx}{dt}&= ax - bxy \\
                \frac{dy}{dt}&= -cy + dxy - eyz \\
                \frac{dz}{dt}&= -fz + gyz \\
            \end{aligned}\right.
        \end{equation}
    \item $a,b,c,d,e,f,g > 0$, $a,b,c,d$ are like in the Lotka volterra
        equations and $e$ is predation of $y$ by $z$, $f$ is the death rate of
        $z$ in absence of prey, $g$ is propagation rate of $z$ if there is prey
    \item populations are $>0$ so we only look at positive octant
\end{itemize}

\subsection{Aanlysis of the model}

\subsubsection{The coordinate planes}

\begin{itemize}
    \item each coordinate plane is invariant with respect to the system
    \item a surface is invariant to a system S if every solution starts on
        S and does not escape S -- if species go extinct, they will not
        reappear
    \item definition of what is means to have an invariant system
    \item following, each of the coordinate planes is invariant
    \item if $z=0$, we have classic L-V equations
    \item if $y=0$ we get exponential growth for $x$ and exponential decline
        for $z$ -- makes biological sense (ignoring the unbouded growth of $x$)
    \item this system has a simply found solution $z = Kx^{-f/a}$
    \item if $x$ does not exist, all species go extinct eventually
    \item in certain cases $z$ may grow at first by eating $y$ but after $y=0$
        they will die out too
\end{itemize}

\subsubsection{Equilibria and linear analysis}

\begin{itemize}
    \item it is often useful to investigate solutions that do not change with
        time, where $x_t = y_t = z_t = 0$ -- called equilibria,
        steady-state-solutions, fixed points
    \item two obvious equilibria: $(0,0,0)$ and $(c/d, a/b,0)$ from the L-V
    \item another special case $a/b=f/g$ yields ray of solutions $(s,
        a/b,(ds-c)/c)$ where $s \ge c/d$
    \item asymptotically stable if points close to equilibrium tend to it
    \item if an equation can be linearized, its stability often depends on the
        stability in the associated linearized system
    \item the behavior at that point is determined by the Jacobian of the
        matrix
    \item if all real parts of the eigenvalues have negative real parts the
        system is asymptotically stable, if not it is not stabel
    \item good tool for the analysis of nonlinear systems near equilibria is
        the Center Manifold Theorem
    \item each equilibrium has manifolds, one stable, one unstable, but
        potentially many center manifolds
    \item for the analysis of the system using the center manifold theorem is
        useful -- might be interesting to use for all three cases to take up
        space and time
\end{itemize}

\subsubsection{The case $ga = fb$}

\begin{itemize}
    \item surfaces $z = Kx^{-f/a}$ might be invariant
    \item on those surfaces are periodic orbits that enclose the fixed points
        on the ray $(s,a/b,(ds-c)/c)$
    \item proof that if $ga=fb$  the aforementioned surfaces are invariant to
        the original system of equations
    \item now the equation can be implicitely be solved on each surface
    \item for a fixed K the system becomes:
        \begin{equation}
            \left\{\begin{aligned}
                x_t &= ax - bxy \\
                y_t &= -cy + dxy - eyKx^{-\frac{f}{a}}
            \end{aligned}\right.
        \end{equation}
    \item from quotients we can find separable equations and solve them
    \item \textbf{read from section (5) onwards to understand}
    \item this completely characterizes the case $ga=fb$ 
    \item all three species have populations that vary over time with common
        periods
    \item peak follow: $x$, then $y$, then $z$
\end{itemize}

\subsubsection{The cases $ga \neq fb$}

First $ga < fb$

\begin{itemize}
    \item all solutions seem to spiral down to the $xy$-plane
        towards a periodic solution
    \item solutions move down across surfaces $z = Kx^{-f/a}$ from higher to
        lower values of $K$
    \item next proof shows that solutions travel down a set and bounded path in
        \textbf{proposition 4}
    \item thus for $ga < fb$ all solutions tend towards the $xy$-plane or $z=0$
    \item thus the apex predator always goes extinct in these circumstances and
        the two lower species exhibit standard L-V behavior
\end{itemize}

Second $ga > fb$\newline

\begin{itemize}
    \item analogous in solution, all trajectories in the positive first octant
        escape to $+\infty$ as $t$ increases -- the trajectories travel up the
        surfaces
    \item this means that populations $x,z$ tend to $\infty$ and that $y$
        experiences larger and larger fluctuations
    \item all are non-monotonic though
\end{itemize}

\subsection{Conclusion and comments}

\begin{itemize}
    \item survival of $z$ only depends on $a,b,f,g$: if $ag<fb$ $z$ dies out,
        if $ag \ge fb$ z survives and grow without bound if $ag > fb$
    \item fits our intuition about larger values of $a,g$ and their advantages
        for $z$
    \item larger values of $b,f$ are inhibitive for $z$
    \item $c,d,e$ who are most directly influencing $y$ don't have an effect on
        whether $z$ goes extinct
    \item $y$ is basically a conduit from $x$ to $y$
    \item $y$ cannot go extinct if $x$ remains
    \item this is an excellent model for learning etc because it features a lot
        of good things that generally don't come up
\end{itemize}


%==============================================================================

\section{Hoppensteadt: Predator-prey model}

\subsection{Introduction}

\begin{itemize}
    \item ppm are the building blocks of bio- and ecosystems
    \item biomasses are grown out of their resource masses
    \item resource-consumer, plant-herbivore, parasite-host, disease-immune
        system, susceptible-infectous ...
    \item general loss-win interactions found outside ecology
\end{itemize}

\subsection{A General Predator-Prey Model}

\begin{itemize}
    \item two populations
    \item size at time $t$ is $x(t), y(t)$, population numbers, concentrations,
        both are continuous
    \item changes of population over time are the time derivatives $\dot x,
        \dot y$, equivalent to $dx/dt, dy/dt$
    \item general system, $f,g$ are the \textit{per capita growth rates} of the
        species, $f_y < 0, g_x > 0$, f is prey, g is predator
        \begin{align}\nonumber
            &\dot x = x f(x,y) \\ \nonumber
            &\dot y = y g(x,y)
        \end{align}
\end{itemize}

\subsection{Lotka-Volterra Model}

\begin{itemize}
    \item 1926 Vito Volterra proposed a differential equation to describe
        increase in predator fish and decrease in prey fish in the Adriatic sea
        during WWI
    \item 1925 in the US Alfred Lotka described a hypothetical chemical
        reaction with oscillating concentrations
    \item Lotka-Volterra model is the simplest predator-prey model
    \item based on linear, per-capita growth rates; prey: $f = b - py$,
        predators: $g = rx - d$; $b$ -- prey ($x$) growth rate with no
        predators, $p$ -- impact of predation on $\dot x$ of prey, $d$ --
        decline of predators with no prey, $r$ -- growth of predator population
        depending on prey numbers
    \item full predator-prey model
        \begin{equation}\nonumber
            \left\{
            \begin{aligned}
                &\dot x = (b - py)x \qquad \text{(Prey)}\\
                &\dot y = (rx - d)y \qquad \text{(Predators)}
            \end{aligned}
            \right.
        \end{equation}
    \item system can be integrated directly, any solution of the system $(x(t),
        y(t))$ satisfies $C=b \ln y(t) - p y(t) - r x(t) + d \ln x(t)$ for all
        $t$
    \item $C = b \ln y(0) - p y(0) - r x(s) + d \ln x(0)$
    \item for phase plots with contour plot: $z = b \ln y - py - rx + d \ln x$
    \item the curves describe solutions of the system -- because the curves are
        closed we have periodic oscillations
    \item \textbf{if $b > 0$ -- prey multiplies on its own} you have $(0,0)$
        and $(d/r,b/p)$ as equilibria, the latter one has single peak in $z$
\end{itemize}

\subsection{Lotka and Volterra}

\subsubsection{Alfred James Lotka}

\begin{itemize}
    \item 1880--1949
    \item chemist, demographer, ecologist, mathematician
    \item born in Lviv, Ukraine (then Lemberg, Austria)
    \item to US in 1902, wrote papers about chemical oscillations, theoretical
        biology
    \item then worked at insurance company
    \item eventually became president on Population Association of America
\end{itemize}

\subsubsection{Vito Volterra}

\begin{itemize}
    \item 1860--1940
    \item mathematician, physicist
    \item born in Italy
    \item attended university of Pisa, wrote a book on integral and
        integro-differential equations
    \item after WWI he returned to applications of maths in biology
    \item joined opposition to Mussolini in 1922, refused oath in 1931 and left
        the country to live abroad
\end{itemize}

\subsection{Kermack-McKendrick Model}

\begin{itemize}
    \item in epidemiology we can use these equations too
    \item prey $\rightarrow$ susceptibles; predators $\rightarrow$ infectives
    \item susceptibles can become infectives and infectives can become
        ineffective
    \item critical value / tipping point: $R \equiv rx(0)/d = 1$ is the tipping
        point
    \item some suceptibles will always survive -- herd immunity
\end{itemize}

\subsection{Jacob-Monod Model}

\begin{itemize}
    \item this model accounts for limited uptake rates (e.g. bacteria)
    \item $x$ is population of feeders, they feed on chemical species of
        concentration $y$
    \item $V$ -- uptake velocity, $K$ -- saturation constant, $Y$ -- yield of
        $x$ per unit of $y$ taken up
        \begin{equation}\nonumber
            \left\{
            \begin{aligned}
                &\dot x = \frac{Vy}{K+y}x \ &\text{(Feeders / species $x$)}\\
                &\dot y = -\frac{1}{Y}\frac{Vy}{K+y}y \ 
                            &\text{(Food / nutrient)}
            \end{aligned}
            \right.
        \end{equation}
    \item if $y = K$ uptake velocity is $V/2$; $y=K$ is taken as a tipping
        point: if $y < K$ the uptake is ignored
    \item this stuff underlines a lot of biology, microbiology, food
        engineering
    \item as $t \rightarrow \infty$ the nutrients are depleted
    \item some of the terms can be replaced by other ones to account for
        different environments
\end{itemize}

\subsection{Logistic Equation}

\begin{itemize}
    \item $V$ and $YK$ can be very large compared to the other data but the
        ratio is of moderate size $V/(KY) \approx r$
        \begin{equation}\nonumber
            \dot x = \frac{V(C-x)}{YK+(C-x)}x
        \end{equation}
    \item if we have $V/(KY) \approx r$ and get $\dot x = r (C-x)x$
    \item this has kind of the same results as Jacob-Monod
\end{itemize}

\subsection{Predation with Time Delays: Chaos in Ricker's Reproduction
Equation}

\begin{itemize}
    \item accurate time delays and stuff
    \item \textbf{goes into too much detail, not relevant}
\end{itemize}

%==============================================================================

\section{Israel: On the contribution of Volterra and Lotka to the development
of modern biomathematics}

\begin{itemize}
    \item birth of modern biomathematics took place in 1920s -- maths not as
        mere aid but as conceptual tool and application of determinist and
        mechanist conceptions to biology
    \item now all of mathematical analysis was used in biology
    \item classical mathematical and physical methods and concepts now used in
        maths
    \item 1925 -- Lotka publishes famous treatise, 1926 Volterra publishes
        first paper on population dynamics:
        \begin{equation}\nonumber
            \left\{\begin{aligned}
                \frac{dx}{dt} &= Ax - Bxy \\
                \frac{dy}{dt} &= Cxy - Dy
            \end{aligned}\right.
        \end{equation}
    \item which of them takes priority? always and important question when
        these things are independently and at the same time discovered, seems
        like is was supposed to happen
    \item 1926 Volterra got published, Lotka had stuff published before and
        written about it in 1920 and 1910
    \item Lotka used an analogy between the chemical system and the biological
        system to derive his predator prey equations which is interesting
    \item only reason that Lotka didn't claim priority based on those previous
        papers is because he did not see the relevant concepts in his paper
    \item both the biology and chemistry examples can be called isomorphisms of
        each other -- they are fundamentally equivalent, both are non-linear
        oscillators
    \item both of their approaches are rooted in empirical analysis as was
        normal back then, they didn't use analogy like one might today
    \item Volterra was in favor the mathematicalization of biology --
        differential equations are the obvious results of this
    \item Volterra started because his son-in-law sent him a paper about fish
        in the Adriatic sea
    \item he took a very physical approach, using friction between members of
        species and energy derivations
\end{itemize}

%==============================================================================

\section{Kingsland: Alfred J. Lotka and the origins of theoretical population
ecology}

\begin{itemize}
    \item ecology has borrowed from many disciplines, but phtsical chemistry is
        one of the more unlikely candidates
    \item somewhat stimulated the development of population ecology
    \item Alfred James Lotka is responsible for this
    \item his real goal was to create \textit{physical biology}, applying
        physical principles to biological systems
    \item Lotka started papers by discussing chemical systems and then moving
        to biological examples
    \item first systems were undamped and later damped 
    \item L-V showed that even just 2 populations could regulate each other
        while ecologists at the time were thinking of 5 species food chains
\end{itemize}

%==============================================================================

\section{Keyszig: Chapter 4}

\subsection{Chapter 4.5: Qualitative Methods for Nonlinear Systems}

\begin{itemize}
    \item qualitative methods are methods of finding qualitative information on
        solutions without actually solving the equations
    \item assumptions: autonomous system, $t$ does not occur explicitly
    \item talk about families of solutions
\end{itemize}

\subsubsection{Lotka-Volterra Population Model -- Example 3}

\begin{itemize}
    \item prey has unlimited food, exponential growth without foxes
    \item prey is killed at a rate proportional to both population numbers
    \item predators die exponentially if there is no prey, if there is prey
        they grow exponentially to both population numbers
    \item typical system
    \item critical points: factor out population numbers and set one of the
        factors to 0, get the linearization and see that this is a saddle
        point, which is %TODO: find K's explanation
    \item second critical point is also linearized, we get a family of ellipses
        that has the critical point at it's center
    \item if the non-linear system is analyzed, we see it has the same center
        but closed trajectories and not ellipses
\end{itemize}

%==============================================================================
\section{Terman: State Space}

\begin{itemize}
    \item state space: set of all possible states of a dynamic system
    \item each state of the system corresponds to a point in state space
    \item in pendulum: pairs of "(angle, velocity)" which forms a cylinder
    \item can be finite -- just some points
    \item finite-dimensional -- infinite number of points forming a smooth
        manifold, for ODEs and mapping -- often called \textbf{phase space}
    \item infinite-dimensional -- for PDEs and delay differential equations
    \item degrees of freedom is the number of variables needed to completely
        describe a system
\end{itemize}

\subsection{Phase portrait}

\begin{itemize}
    \item dynamical things make curves or points in phase space
    \item change of a dynamical system corresponds to a trajectory in phase
        space
    \item set of all trajectories forms the phase portrait
    \item because actual solution of nonlinear equations are often not possible
        phase portraits are often used to study them
\end{itemize}

\subsection{Phase Line}

\begin{itemize}
    \item if the system can be described by one variable we have this case
    \item line is partitioned by equilibria etc.
    \item one can study the equation simply with this line
\end{itemize}

\subsection{Phase plane}

\begin{itemize}
    \item arise in 2D autonomous ODEs, like the L-V equations
    \item if $(x(t), y(t))$ is a solution to the system, then at $t=p$,
        $(x(p),y(p))$ is a point in phase plane, point changes over time and
        traces a trajectory in the phase plane
    \item solution trajectories have their velocity vector given, vector field
        assigns stuff, see p. 3 beginnig
    \item equilibrium points are where both $f$ and $g$ are 0, these points are
        constant with respect to time
    \item some elements or solutions are periodic and have periods of time when
        they repeat -- they are closed curves in the phase plane
    \item periodic solutions are stable if solutions near it remain near it
        through time
\end{itemize}

\subsection{Higher Dimensional and Abstract state spaces}

\begin{itemize}
    \item n-dimensional explanation
    \item abstract spaces
    \item random examples and pendulum example
\end{itemize}

%==============================================================================

\section{Wangersky: Lotka-Volterra population models}

\subsection {Introduction}

\begin{itemize}
    \item tension between theorists and experimentalists
    \item math. modeling in biology is often neglected because both sides don't
        really know about the other and thus make errors
    \item modeling is misunderstood, only see as predictors and called useless
        if they aren't 100\% accurate
    \item only very special models have any chance of being accurate -- normal
        models are used to find forms of solutions and not future states of
        systems
    \item descriptive model: best-fit curves based on data, simple and good
        fit
    \item analytical model: considers mechanisms of the system, mostly logic,
        little data, very complex, can predict with changing circumstances,
        theoretical ecology, often based on assumptions
    \item most models are a mix, problem: fitting constants might be
        misinterpreted
    \item assumptions are made, sometimes unaware
\end{itemize}

\subsection{Growth of a single species population}

\begin{itemize}
    \item exponential growth of species like $dx/dt=rx$, $x$ population, $r$
        constant of growth
    \item not best assumption but ok in labs, if just taken to fit data it's
        ok, if we take it as birthrate - deathrate and take is as intrinsic
        rate of natural increase that's not ok
    \item can add a proportion of the max growth rate that is achieved
    \item population density tends to be a factor too, it's ignored here
    \item if it's incorporated, we get $dx/dt-rx[1-(x/K)]$, $K$ is number of
        supported organisms in an environment
    \item today logistic equations are generally used for this
    \item model only works for species of long or short lifetimes where changes
        in environment are negligible 
    \item oscillations often occur, even in lab conditions where there is no
        scarcity -- L-V models have that
    \item time lag is generally introduced here to fit reality better
\end{itemize}

\subsection{Prey-Predator Equations}

\begin{itemize}
    \item typical model, $x$ number of prey, $y$ number of predators, $a$ prey
        mortality liked to predator and prey numbers, $b$ predator growth
        linked to predator and prey numbers, $d$ mortality constant of
        predators
    \item models and their periodic behavior is well known, but few examples of
        oscillations are known that conclusively depends on predator prey
        cycles -- not even snowshoe hare-lynx in (50) and brown lemming-grass
        (206)
    \item there are some examples of well-fitting lab populations, mostly
        bacteria
    \item first step is to introduce dampening into prey population and then
        time delay to the predators
    \item \textbf{mostly irrelevant stuff}
\end{itemize}

%==============================================================================


\nocite{*}
\bibliographystyle{siam}   %abbrv
\bibliography{../bibliography}

\end{document}
