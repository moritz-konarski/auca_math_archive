% Presentation on Predator-Prey models
% For 07.05.2020, Moritz Konarski, AUCA

\documentclass[hyperref={colorlinks,allcolors=black}]{beamer}

\mode<presentation>
{
  \usetheme{Goettingen}
  \usecolortheme{seahorse}
  \usefonttheme{serif}
  \setbeamertemplate{navigation symbols}{}
  \setbeamertemplate{caption}[numbered]
} 

\usepackage[english]{babel}
\usepackage[utf8]{inputenc}
\usepackage[T1]{fontenc}
\usepackage{amsmath}
\usepackage{amsfonts}
\usepackage{textcomp}
\usepackage{graphicx}
    \graphicspath{{../../graphics/}}
\usepackage{subfig}
\usepackage{wrapfig}
\usepackage{float}
\usepackage{caption}

\newcommand{\figref}[1]{\figurename~\ref{#1}}

\newcommand{\makefig}[4]{
    \begin{figure}[#1]
        \captionsetup{justification=centering}
        \includegraphics[width=#2]{#3}
        \caption{#4}
        \label{fig:#3}
    \end{figure}
}

\newcommand{\makewrapfig}[5]{
    \begin{wrapfigure}{#1}{#2}
        \captionsetup{margin=10pt,justification=centering}
        \includegraphics[width=#3]{#4}
        \caption{#5}
        \label{fig:#4}
    \end{wrapfigure}
}

\AtBeginSection[]{
    \begin{frame}
        \vfill
        \centering
        \begin{beamercolorbox}[sep=8pt,center,shadow=false,rounded=false]{title}
            \usebeamerfont{title}\insertsectionhead\par
        \end{beamercolorbox}
        \vfill
    \end{frame}
}

\setbeamertemplate{frametitle}[default][center]

\setbeamertemplate{bibliography item}{\insertbiblabel}

%==============================================================================

\title[Predator-Prey Equations]{Predator-Prey Equations:\\Modeling Food Chains}

\author[M. Konarski]{Moritz M. Konarski}

\institute[AUCA]{Applied Mathematics Department \\ 
    American University of Central Asia}

\date{\today}

%==============================================================================

\begin{document}

\begin{frame}
  \titlepage
\end{frame}

\begin{frame}{Outline}
  \tableofcontents
\end{frame}

%==============================================================================

\section{Introduction}

\begin{frame}{Background}
    \begin{itemize}\setlength\itemsep{1em}
        \item predator-prey equations are differential equations describing 
            populations of predators and prey
        \item the most famous ones are Lotka-Volterra equations
        \item independently derived by Alfred J. Lotka (1880--1949) and Vito 
            Volterra (1860--1940) in the 1920s
        \item Volterra observed fish, Lotka chemical reactions and got the same
            equations -- both are the same system \cite{hoppensteadt}
    \end{itemize}
\end{frame}

\subsection{Equations}

\begin{frame}{Lotka-Volterra Equations}
    \begin{itemize}\setlength\itemsep{1em}
        \item simplest predator-prey equations
        \item describe interactions of one predator and one prey
    \end{itemize}
%
    \begin{equation}    
        \left\{\begin{aligned}
            &\frac{dx}{dt} = ax - bxy \quad &\text{Prey}\\
            &\frac{dy}{dt} = -cy +dxy \quad &\text{Predator}
        \end{aligned}\right.
        \label{eq:2s_system}
    \end{equation}
%
    \begin{itemize}\setlength\itemsep{1em}
        \item parameters $a,b,c,d>0$, following \cite{chauvet}
    \end{itemize}
\end{frame}

\begin{frame}{Expanded Lotka-Volterra Equations}
    \begin{itemize}\setlength\itemsep{1em}
        \item expansion of \eqref{eq:2s_system} to include another predator
        \item interactions: $y$ eats $x$ and $z$ eats $y$
    \end{itemize}
%   
    \begin{equation}
        \left\{\begin{aligned}
            &\frac{dx}{dt} = ax-bxy            &\text{Prey}\\
            &\frac{dy}{dt} = -cy+dxy-eyz \quad &\text{Intermediate Predator}\\
            &\frac{dz}{dt} = -fz+gyz           &\text{Apex Predator}
        \end{aligned}\right.
        \label{eq:3s_system}
    \end{equation}
%
    \begin{itemize}\setlength\itemsep{1em}
        \item parameters $a,b,c,d,e,f,g>0$, following \cite{chauvet}
    \end{itemize}
\end{frame}

\subsection{Phase Planes}

\begin{frame}{Phase Planes}
    \begin{itemize}\setlength\itemsep{1em}
        \item space where all points are states of a system, e.g. system
            \eqref{eq:2s_system}
        \item moving points form trajectories (lines), closed trajectories are
            periodic solutions 
        \item stationary points are equilibria, don't change over time
        \item give qualitative insights into equation without solving
            \cite{terman}
    \end{itemize}
\end{frame}

\begin{frame}{Phase Planes - Example}
    \centering
    $x'=y$ (angle) and $y'=-\sin(x)$ (angular velocity) 
    \makefig{h}{0.7\textheight}{example_phase_plane}
        {Phase plane of a pendulum with contour lines}
\end{frame}

%==============================================================================

\section{Two-Species Food Chain}

\subsection{General Behavior}

\begin{frame}{Equations}
    \begin{itemize}\setlength\itemsep{1em}
        \item standard Lotka-Volterra equations
        \item equivalent to system \eqref{eq:3s_system} with $z=0$
    \end{itemize}
%
    \begin{equation}
        \left\{\begin{aligned}
            &\frac{dx}{dt} = x(1 - y) \quad &\text{Prey}\\
            &\frac{dy}{dt} = y(x - 1) \quad &\text{Predator}
        \end{aligned}\right.
        \label{eq:2s_example}
    \end{equation}
%
    \begin{itemize}\setlength\itemsep{1em}
            \item from \eqref{eq:2s_system}, parameters $a=b=c=d=1$ chosen for 
                simplicity
    \end{itemize}
\end{frame}

\begin{frame}{Phase Plane}
    \begin{figure}[H]%
        \subfloat[Phase plane]{
            {\includegraphics[width=0.46\textwidth]{2s_phase_plane}}}%
        \quad
        \subfloat[Normalized phase plane]{
            {\includegraphics[width=0.46\textwidth]{2s_phase_plane_normalized}}}%
        \captionsetup{justification=centering}
        \caption{Two-species system phase planes with $a=b=c=d=1$}%
        \label{fig:2s_phase_plane}%
    \end{figure}
\end{frame}

\begin{frame}{Phase Plane with Contours}
    \begin{itemize}
        \item solutions to \eqref{eq:2s_system} have the form  
            \eqref{eq:2s_solution}
    \end{itemize}
%
    \begin{equation}
        C = a \ln y - by + c \ln x - dx
        \label{eq:2s_solution}
    \end{equation}
%
    \begin{itemize}
        \item for $a=b=c=d=1$ in \eqref{eq:2s_example} we get
    \end{itemize}
%
    \begin{equation}\nonumber
        C = \ln y - y + \ln x - x
    \end{equation}
%
    \begin{itemize}
        \item this can be used to graph solutions of \eqref{eq:2s_example} in
            a phase plane
    \end{itemize}
\end{frame}

\begin{frame}{Phase Plane with Contours}
    \makefig{H}{0.7\textheight}{2s_phase_plane_contours}
        {Two-species system phase plane with normalized vectors, contour lines, 
        and $a=b=c=d=1$}
\end{frame}

\subsection{Special Cases}

\begin{frame}{Case $x=0$ -- No Prey}
    \makefig{H}{0.85\textwidth}{2s_x_zero_graph}
        {Two-species system graph for $x=0$, $y=3$, and $a=b=c=d=1$}
\end{frame}

\begin{frame}{Case $y=0$ -- No Predators}
    \makefig{h}{0.85\textwidth}{2s_y_zero_graph}
        {Two-species system graph for $x=6$, $y=0$, and $a=b=c=d=1$}
\end{frame}

\begin{frame}{Case $x=6$, $y=3$ -- Contour}
    \makefig{h}{0.7\textwidth}{2s_contour}
        {Two-species system contour for $x=6$, $y=3$, and $a=b=c=d=1$}
\end{frame}

\begin{frame}{Case $x=6$, $y=3$ -- Graph}
    \makefig{h}{0.9\textwidth}{2s_graph}
        {Two-species system graph for $x=6$, $y=3$, and $a=b=c=d=1$}
\end{frame}

%==============================================================================

\section{Three-Species Food Chain}

\begin{frame}{Equations}
    \begin{itemize}\setlength\itemsep{1em}
        \item equation \eqref{eq:3s_system} with parameters chosen for
            simplicity
    \end{itemize}
%
    \begin{equation}
        \left\{\begin{aligned}
            &\frac{dx}{dt} = x(1-y)          &\text{Prey}\\
            &\frac{dy}{dt} = y(x-z-1) \qquad &\text{Intermediate Predator}\\
            &\frac{dz}{dt} = z(y-1)          &\text{Apex Predator}
        \end{aligned}\right.
        \label{eq:3s_example}
    \end{equation}
%
    \begin{itemize}\setlength\itemsep{1em}
        \item $a=b=c=d=e=f=g=1$ as parameters
    \end{itemize}
\end{frame}

\subsection{Coordinate Planes}

\begin{frame}{Case $z=0$}
    No apex predators -- becomes the same system as the two-species system
    \eqref{eq:2s_example}
    \begin{equation}\nonumber
        \left\{\begin{aligned}
            &\frac{dx}{dt}=x(1-y)                &\text{Prey}\\
            &\frac{dy}{dt}=y(x-0-1)=y(x-1)\quad &\text{Intermediate Predator}\\
            &\frac{dz}{dt}=0(y-1)=0              &\text{Apex Predator}
        \end{aligned}\right.
    \end{equation}
\end{frame}

\begin{frame}{Case $z=0$ -- Contour}
    \makefig{h}{0.7\textwidth}{3s_z_zero_contour}
        {Three-species system contour for $x=6$, $y=3$, $z=0$, 
        $a=b=c=d=e=f=g=1$}
\end{frame}

\begin{frame}{Case $z=0$ -- Graph}
    \makefig{h}{0.8\textwidth}{3s_z_zero_graph}
        {Three-species system graph for $x=6$, $y=3$, $z=0$, $a=b=c=d=e=f=g=1$}
\end{frame}

\begin{frame}{Case $x=0$}
    If $x=0$ there is no prey -- other species starve
    \begin{equation}\nonumber
        \left\{\begin{aligned}
            &\frac{dx}{dt}=0(1-y)=0              &\text{Prey}\\
            &\frac{dy}{dt}=y(0-z-1)=y(-z-1)\quad &\text{Intermediate Predator}\\
            &\frac{dz}{dt}=z(y-1)                &\text{Apex Predator}
        \end{aligned}\right.
    \end{equation}
\end{frame}

\begin{frame}{Case $x=0$ -- Graph}
    \makefig{h}{0.9\textwidth}{3s_x_zero_graph}
        {Three-species system graph for $x=0$, $y=3$, $z=1$, $a=b=c=d=e=f=g=1$}
\end{frame}

\begin{frame}{Case $y=0$}
    Without intermediate predators $x$ grows and $z$ starves
    \begin{equation}\nonumber
        \left\{\begin{aligned}
            &\frac{dx}{dt}=x(1-0)=x         &\text{Prey}\\
            &\frac{dy}{dt}=0(x-z-1)=0\qquad &\text{Intermediate Predator}\\
            &\frac{dz}{dt}=z(0-1)=-z        &\text{Apex Predator}
        \end{aligned}\right.
    \end{equation}
\end{frame}

\begin{frame}{Case $y=0$ Graph}
\makefig{h}{0.9\textwidth}{3s_y_zero_graph}{Three-species system graph for $x=6$, $y=0$, $z=1$, $a=b=c=d=e=f=g=1$}
\end{frame}

\begin{frame}{Further Analysis; Case $ga>fb$}
\begin{itemize}
\setlength\itemsep{1em}
    \item in \cite{chauvet} the authors use further criteria to
        classify \eqref{eq:3s_system}
\begin{equation}\nonumber
    ga > fb, \qquad ga < gb, \qquad  ga = fb.
\end{equation}
\item For example $a=g=1.1$ and all other constants 1
\end{itemize}
\begin{equation}\nonumber
    \left\{\begin{aligned}
        &\frac{dx}{dt} = 1.1x - xy              &\text{Prey,}\\
        &\frac{dy}{dt} = -y + xy - yz \quad &\text{Intermediate Predator,}\\
        &\frac{dz}{dt} = -z + 1.1yz             &\text{Apex Predator.}
    \end{aligned}\right.
\end{equation}
\end{frame}

\begin{frame}{Case $ga>fb$ Graph}
\makefig{h}{0.9\textwidth}{3s_ga_g_fb_graph}{Three-species system graph for $x=6$, $y=3$, $z=1$, $b=c=d=e=f=1$, and $a=g=1.1$}
\end{frame}

\begin{frame}{Case $ga < fb$}
For example $b=f=1.1$ and all other constants equal 1
\begin{equation}\nonumber
    \left\{\begin{aligned}
        &\frac{dx}{dt} = x - 1.1xy              &\text{Prey,}\\
        &\frac{dy}{dt} = -y + xy - yz \quad &\text{Intermediate Predator,}\\
        &\frac{dz}{dt} = -1.1z + yz             &\text{Apex Predator.}
    \end{aligned}\right.
\end{equation}
\end{frame}

\begin{frame}{Case $ga < fb$ Graph}
\makefig{h}{0.9\textwidth}{3s_ga_l_fb_graph}{Three-species system graph for $x=6$, $y=3$, $z=1$, $a=c=d=e=g=1$, and $b=f=1.1$}
\end{frame}

\begin{frame}{Case $ga < fb$ Contour}
\makefig{h}{0.65\textwidth}{3s_ga_l_fb_contour}{Three-species system contour for
$x=6$, $y=3$, $z=1$, $a=c=d=e=g=1$, and $b=f=1.1$}
\end{frame}

\begin{frame}{Case $ga=fb$}
\begin{itemize}
\setlength\itemsep{1em}
    \item original equation \eqref{eq:3s_example} does not change
    \item all constants equal 1, $a=b=c=d=e=f=g=1$
    \item system could be periodic
\end{itemize}
\end{frame}

\begin{frame}{Case $ga=fb$ Graph}
\makefig{h}{0.9\textwidth}{3s_ga_eq_fb_graph}{Three-species system graph for
$x=6$, $y=3$, $z=1$, and $a=b=c=d=e=f=g=1$}
\end{frame}

\begin{frame}{Case $ga=fb$ Contour}
\makefig{h}{0.65\textwidth}{3s_ga_eq_fb_contour}{Three-species system contour for
$x=6$, $y=3$, $z=1$, and $a=b=c=d=e=f=g=1$}
\end{frame}

%==============================================================================

\section{Conclusion}

\begin{frame}{Conclusion}
    \begin{itemize}\setlength\itemsep{1em}
        \item models fit general intuition
        \item inaccuracies: unlimited growth, only one species as prey
        \item equation had great impact on ecology \cite{wangersky}
        \item Lotka-Volterra equations \eqref{eq:2s_system} are among the most
            famous differential equations
    \end{itemize}
\end{frame}

%==============================================================================

\section*{Q and A}

%==============================================================================

\begin{frame}{References}
    \begin{thebibliography}{6}\scriptsize

        \bibitem{chauvet} 
            \textsc{E. Chauvet, J. E. Paullet, J. P. Previte, and Z. Walls},
            \textit{Lotka-Volterra three-species food chain}, Mathematics
            Magazine, 75 (2002), pp. 243–255. Retrieved from:
            \url{http://www.jstor.org/stable/3219158}

        \bibitem{hoppensteadt}  
            \textsc{F. Hoppensteadt}, \textit{Predator-prey model},
            Scholarpedia, 1 (2006), p. 1563. Retrieved from:
            \url{http://www.scholarpedia.org/article/Predator-prey_model}.

        \bibitem{israel} 
            \textsc{G. Israel}, \textit{On the contribution of Volterra and
            Lotka to the development of modern biomathematics}, History and
            Philosophy of the Life Sciences, 10 (1988), pp. 37–49. Retrieved
            from: \url{http://www.jstor.org/stable/23328998}.

        \bibitem{kreyszig} 
            \textsc{E. Kreyszig}, \textit{Advanced Engineering Mathematics},
            John Wiley \& Sons, Inc., 10 ed., 2011.

        \bibitem{terman} 
            \textsc{D. H. Terman and E. M. Izhikevich}, \textit{State space},
            Scholarpedia, 3 (2008), p. 1924. Retrieved from:
            \url{http://www.scholarpedia.org/article/Phase_space}.

        \bibitem{wangersky} 
            \textsc{P. J. Wangersky}, \textit{Lotka-Volterra population
            models}, Annual Review of Ecology and Systematics, 9 (1978), pp.
            189–218. Retrieved from:
            \url{https://www.jstor.org/stable/2096748}.

    \end{thebibliography}
\end{frame}
\end{document}
