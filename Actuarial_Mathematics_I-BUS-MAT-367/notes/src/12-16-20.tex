\documentclass[../00_main.tex]{subfiles}

\begin{document}

\section{Notes 16.12.2020 -- Bonds}

\subsection{Introduction}

\begin{itemize}
    \item bonds are low-risk growth opportunities
    \item bonds promise payments at certain dates (coupon payments)
    \item at the redemption date the last of the coupon payments occurs
    \item issue date is the date the bond is bought
    \item the term is the time from the issue date until the maturity date
    \item a bond is noncallable if the maturity date is fixed
    \item if a bond only has a payment at the maturity date it is simply a loan
        (called zero-coupon bonds or pure discount bonds)
    \item coupon bonds have payments prior to the redemption date
    \item coupon payments tend to be level and evenly spaced, the redemption
        payment tends to be larger
\end{itemize}

\subsection{Alphabet soup and basic price formula}

\begin{itemize}
    \item $F$ --- face/par value of the bond only used for coupon payments (is
        often equal to the redemption value)
    \item $Fr=\frac{F\alpha}{m}$ --- coupon rate, paid $m$ times per year based
        on convertible rate $\alpha$, $r$ is the effective rate
    \item $n$ --- number of coupon periods in the bond term, $N$ is the number
        of years, so $n = Nm$
    \item $C$ --- redemption amount paid at the end of $n$ periods
    \item if $F=C$, the bond is redeemable-at-par or a par-value bond; if $C$
        is not given, $F=C$ should be assumed
    \item $g = \frac{Fr}{C}$ --- modified coupon rate, expresses the coupon
        rate in terms of $C$, $Fr = Cg$
    \item $i$ --- annual effective yield rate, $j$ --- investors effective
        yield rate per period
        \begin{equation}\nonumber
            i = (1 + j)^m - 1
        \end{equation}
    \item $I$ --- nominal yield rate convertible $m$ times per year
        \begin{equation}\nonumber
            j = \frac{I}{m} \qquad i = \left(1 + \frac{I}{m}\right)^m - 1
        \end{equation}
    \item $G = \frac{Fr}{j}$ --- base amount, expresses the coupon amount in 
        terms of yield rate $j$, then $Fr = Cg = Gj$
    \item $P$ --- price the investor paid to get yield rate $j$
    \item $v_j = (1 + j)^{-1}$ --- find previous value 
    \item $K = Cv_j^n$ --- value of the redemption at issue date assuming
        compound interest
    \item the basic price formula is 
        \begin{equation}\nonumber
            P = (Fr) \ax{\angln j} + Cv_j^n = (Fr) \ax{\angln j} + K
        \end{equation}
    \item 
\end{itemize}

\subsection{The premium-discount formula}

\begin{itemize}
    \item because we have 
        \begin{equation}\nonumber
            \ax{\angln j} = \frac{1 - v_j^n}{j}
        \end{equation}
        and
        \begin{equation}\nonumber
            v_j^n = 1 - j \ax{\angln j}
        \end{equation}
        we can write the price formula as 
        \begin{equation}\nonumber
            P = (Fr) \ax{\angln j} + Cv_j^n = (Cg) \ax{\angln j} 
                + C (1 - j \ax{\angln j})
        \end{equation}
        and finally 
        \begin{equation}\nonumber
            P = C(g - j) \ax{\angln j} + C
        \end{equation}
    \item this is called the premium-discount formula for a bond
    \item bonds sell at a premium if their price is higher than their
        redemption amount, and the premium is 
        \begin{equation}\nonumber
            P = C(g - j) \ax{\angln j}
        \end{equation}
    \item if the price is lower than the redemption amount, the discount is
        \begin{equation}\nonumber
            P = C(j - g) \ax{\angln j}
        \end{equation}
\end{itemize}

\subsection{Other pricing formulas}

\begin{itemize}
    \item the base amount formula is 
        \begin{equation}\nonumber
            P = (C - G) v_j^n + G 
        \end{equation}
    \item Makeham's formula is useful when K is known but the number of
        periods $n$ is not
        \begin{equation}\nonumber
            P = \frac{g}{j} (C - K) + K
        \end{equation}
    \item 
\end{itemize}

\subsection{Bond amortization}

\begin{itemize}
    \item because bond payments consist of interest and principal, we can make
        an amortization schedule
    \item we introduce $B_t$ as the balance of the debt at time $t$, depending
        on the yield rate $j$
    \item $B_0 = P$ and $B_n = C$, $B_t$ is called the book value at time $t$
    \item the interest due is $I_t = jB_{t-1}$
    \item the amount of principal adjustment is $P_t = B_{t-1} - B_t$
    \item to find $B_t$ the basic price formula or the premium-discount formula
        may be used
        \begin{equation}\nonumber
            B_t = (Fr) \ax{\angl{n-t} j} + Cv_j^{n-t}
        \end{equation}
        or for the premium-discount formula
        \begin{equation}\nonumber
            B_t = C(g-j) \ax{\angl{n-t} j} + C
        \end{equation}
    \item using the premium-discount formula we can find
        \begin{equation}\nonumber
            I_t = Cg - C(g - j)v_j^{n-t+1}
        \end{equation}
        and
        \begin{equation}\nonumber
            P_t = C(g - j)v_j^{n-t+1}
        \end{equation}
        and we see that
        \begin{equation}\nonumber
            P_t + I_t = Cg
        \end{equation}
    \item we also get a nice recursion formula
        \begin{equation}\nonumber
            B_t = (1 + j) B_{t-1} - Cg
        \end{equation}
\end{itemize}

\subsection{Valuing a bond after its date of issue}

\begin{itemize}
    \item the dirty value of a bond $D_T$ (called dirty because it is
        discontinuous at each coupon date)
        \begin{equation}\nonumber
            D_T = (1 + j)^f B_{\floor*{T}}, \quad f = T - \floor*{T}
        \end{equation}
    \item practical dirty values are more useful because they are continuous
        \begin{equation}\nonumber
            D_T^\text{prac} = (1 + fj) B_{\floor*{T}}, \quad f = T - \floor*{T}
        \end{equation}
    \item we also have practical clean values that are the best
        \begin{equation}\nonumber
            C_T^\text{prac} = B_{\floor*{T}} + f(B_{\floor*{T}+1}
            - B_{\floor*{T}}), \quad f = T - \floor*{T}
        \end{equation}
    \item see page 287 (PDF) of the maths for actuaries book
\end{itemize}

\subsection{Callable bonds}

\begin{itemize}
    \item callable bonds are issued with call provision (agreement) that allow
        the issuer to repay it earlier
    \item the bond can be repaid at any one of the designated call dates
    \item the period before the first call date is called the lockout period
    \item for the investor callable bonds are worse because they are more
        difficult to judge and less safe
    \item to make up for this disadvantage, callable bonds generally have
        higher yield rates, and the redemption prices before maturity are often
        higher 
    \item when considering such a bond, it is wise to look at all the possible
        redemption values to figure out which is the possible worst and focus
        on that
    \item if a bond is bought at a discount, the worst yield rate will be
        attained at the redemption date
    \item if a bond is bought at a premium, the worst yield rate is attained
        when the bond is called at the earliest possible time
\end{itemize}

\section{Interest Rate Sensitivity}

\subsection{Overview}

\begin{itemize}
    \item interest rates are volatile and you can never be sure what you'll get
    \item thus you need to take care -- immunizing your position means making
        sure that your position is good regardless of whether interest goes up
        or down
    \item for example, investing half in bonds that are good if interest is low
        and the other half in bonds that are good if interest is high can
        immunize one somewhat from changing rates
    \item duration of a cashflow is a measure of how sensitive the price is to
        changes in the interest rate
    \item Macaulay duration is a weighted average of the times of the cashflows
\end{itemize}

\subsection{Overview}

\begin{itemize}
    \item assuming a number of cash flows, the price of all of then to get
        a yield rare $i$ is
        \begin{equation}\nonumber
            P(i) = \sum_{t \ge 0} C_t(1+i)^{-t}
        \end{equation}
    \item $i_0$ is the initial interest rate
    \item the Taylor polynomial for this sum is
        \begin{equation}\nonumber
            \sum_{n=0}^\infty \frac{P^{(n)}(i_0)}{n!}(i - i_0)^n = 
                P(i_0) + P'(i_0)(i - i_0) + \frac{P''}{2}(i_0)(i - i_0)^2 
                + \ldots
        \end{equation}
    \item the first two terms give the tangent line approximation to $P(i)$ at
        rate $i_0$
        \begin{equation}\nonumber
            P(i) \approx P(i_0) + P'(i_0)(i - i_0)
        \end{equation}
    \item the second-Taylor-polynomial approximation is
        \begin{equation}\nonumber
            P(i) \approx P(i_0) + P'(i_0)(i - i_0) + 
            \frac{P''}{2}(i_0)(i - i_0)^2 
        \end{equation}
    \item for all these equations the derivates can be found with
        \begin{equation}\nonumber
            P'(i) = - \sum_{t \ge 0} C_t (1 + i)^{-t -1}
        \end{equation}
        and
        \begin{equation}\nonumber
            P''(i) = - \sum_{t \ge 0} C_t t (t + 1) (1 + i)^{-t -2}
        \end{equation}
    \item we have another approximation
        \begin{equation}\nonumber
            \frac{P(i) - P(i_0)}{P(i_0)} \approx \frac{P'(i_0)}{P(i_0)}(i
                - i_0)
        \end{equation}
    \item this allows us to estimate the fractional price increase
    \item thus we can talk about basis points, 100 of which correspond to 
        $0.01 = 1\%$ in increased value ($i = q\% = 0.01q \rightarrow i = 
        (q+i)\% = 0.01(q+1)$)
    \item If the yield is initially equal to $i_0$ and then it increases by one
        hundred basis points, the approximate relative price change is
        $\frac{P'(i_0)}{P(i_0)}$ percent.
    \item we can now define the modified duration
        \begin{equation}\nonumber
            D(i, 1) = - \frac{P'(i)}{P(i)} = 
            \frac{\sum_{t \ge 0} C_t t (1+i)^{-t-1}}
                {\sum_{t \ge 0} C_t (1+i)^{-t}}
        \end{equation}
    \item general modified duration for $i$ convertible $m$ times is
        \begin{equation}\nonumber
            D(i, m) = \left( \frac{1 + i}{1 + \frac{i^{(m)}}{m}}\right) D(i, 1)
        \end{equation}
    \item the Macaulay duration is
        \begin{equation}\nonumber
            D(i, \infty) = D(i,m) \left( 1 + \frac{i^{(m)}}{m}\right)
        \end{equation}
        or alternatively
        \begin{equation}\nonumber
            D(i, \infty) = \sum_{t \ge 0} 
                \left( \frac{C_t (1+i)^{-t}}{P(i)}\right) t
        \end{equation}
    \item for a zero-coupon bond, $D(i, \infty) = N$ and $D(i, 1)
        = \frac{N}{1+i}$
\end{itemize}

\subsection{Convexity}

\begin{itemize}
    \item while tangent approximation works well for small interest changes,
        larger changes can be modeled by the quadratic approximation
    \item thus we have the modified convexity as
        \begin{equation}\nonumber
            C(i, 1) = \frac{P''(i)}{P(i)}
        \end{equation}
    \item this allows us to rewrite our estimation as
        \begin{equation}\nonumber
            \frac{P(i) - P(i_0)}{P(i_0)} \approx -D(i_0,1)(i - i_0) + C(i_0,
            1)\frac{(i-i_0)^2}{2}
        \end{equation}
    \item this indicates that if two investments have the same modified
        duration, the one with the larger absolute convexity is more
        susceptible to interest rate changes
    \item the Macaulay convexity is given by
        \begin{equation}\nonumber
            C(i, \infty) = \sum_{t \ge 0} \left( \frac{C_t (1+i)^{-t}}
                {P(i)}\right)t^2
        \end{equation}
    \item from the Macaulay formulas we know that
        \begin{equation}\nonumber
            C(i, m) = \frac{C(i, \infty) + \frac{1}{m}D(i, \infty)}{\left(
            1 + \frac{i^{(m)}}{m}\right)^2}
        \end{equation}
\end{itemize}

\end{document}
