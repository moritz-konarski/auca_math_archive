\documentclass[../00_main.tex]{subfiles}

\begin{document}

\section{Class 15.10.2020}

\subsection{Amortization}

\begin{itemize}
    \item repay a loan by the \textit{amortization method} -- installment
        payments at periodic intervals
    \item knowing the outstanding principal is important because you need to
        know how much you owe
    \item \textit{prospective method}: outstanding principal is the present
        value of the outstanding payments at that time
    \item \textit{retrospective method}: original principal accumulated until
        then minus the accumulated value of all the payments made until then
    \item this means that we either need to find $\ax{\angln}$ or
        "$\text{original principal }\sx{\angln} -
        \text{payments }\sx{\angln}$"
\end{itemize}

\subsection{Amortization Schedules}

\begin{itemize}
    \item a payment $X$ can be divided into its principal and interest parts
        like so:
        \begin{enumerate}
            \item know or find the outstanding principal 1 time interval before
                $X$, let's call it $P$
            \item the interest portion of $X$ is $iP$
            \item the principal portion of $X$ is $X - iP$
        \end{enumerate}
    \item if a loan in paid back in equal payments of $X$ for $n$ years, the
        interest part of the $k$th payment is
        \begin{equation}\nonumber
            X(1-v^{n-k+1})
        \end{equation}
    \item the principal part of the $k$th payment is
        \begin{equation}\nonumber
            Xv^{n-k+1}
        \end{equation}
    \item an amortization schedule is simply a table showing the payments and
        how they are made up
\end{itemize}

\begin{center}
\begin{tabular}{|r|r|r|r|r|}\hline
    Duration & Payment & Interest & Principal & Outstanding 
    \\ &&& Repaid & Principal \\\hline\hline
    0  &     &        &        & 1,000.00 \\
    1  & 150 & 110.00 &  40.00 &  960.00  \\
    2  & 150 & 105.60 &  44.40 &  915.60  \\
    3  & 150 & 100.72 &  49.28 &  866.32  \\
    $\vdots$  & 150    & $\vdots$ & $\vdots$ & $\vdots$ \\
    12 & 150 & 23.93  &  126.07 &  91.51  \\
    13 & 101.58 & 10.07 &  91.51 &  0.00  \\\hline
\end{tabular}
\end{center}

\end{document}
