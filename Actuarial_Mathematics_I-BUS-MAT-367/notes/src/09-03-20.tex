\documentclass[../00_main.tex]{subfiles}

\begin{document}

\section{Class 03.09.2020}

\subsection{Compound Interest}

\begin{itemize}
    \item most important special case
    \item effective interest rate is fixed
    \item interest earns interest itself
    \item because the interest affects itself, the function is exponential
        \begin{equation}\nonumber
            a(t) = (1+i)^t, \quad t \ge 0
        \end{equation}
    \item amount function for compound interest is
        \begin{equation}\nonumber
            A(t) = k(1+i)^t
        \end{equation}
    \item the effective interest rate for compound interest is constant
    \item what values to choose for $t$ is done like with simple interest,
        either \textit{exact} or \textit{ordinary}
    \item if we want to find some value between integers, we linearly
        interpolate it 
        \begin{equation}\nonumber
            A(t) = A(\floor*{t}) + (t - \floor*{t}) \cdot 
                (A(\ceil*{t})-A(\floor*{t}))
        \end{equation}
    \item to find the time it takes a principal to accumulate to a certain
        value, use logs
        \begin{equation}\nonumber
            t = \frac{\log{(\frac{\text{future value}}{\text{principal}})}}
                {\log{(1+i)}}
        \end{equation}
    \item compound and simple interest graphs only intersect at $(0,1)$ and at
        $(1, 1+i)$, this furthermore gives two cases
        \begin{equation}\nonumber
            \begin{cases}
                &\text{simple i.} > \text{compound i.} \quad \text{for} \quad
                    0 < t < 1 \\
                &\text{compound i.} > \text{simple i.} \quad \text{for} \quad
                    t > 1 \\
            \end{cases}
        \end{equation}
\end{itemize}

\end{document}
