\documentclass[../00_main.tex]{subfiles}

\begin{document}

\section{Class 24.09.2020}

\subsection{Arithmetic and geometric sequences}

\subsubsection{Arithmetic sequences}

\begin{itemize}
    \item $a, a+d, a+2d, a+3d$
    \item $n$th term: $a_n = a + (n-1)d$
    \item sum of first $n$ terms: $S_n = \frac{n}{2}\left[ 2a + (n-1)d\right]$
\end{itemize}

\subsubsection{Geometric sequences}

\begin{itemize}
    \item $a, ar, ar^2, ar^3$
    \item $n$th term: $a_n = ar^{n-1}$
    \item sum of first $n$ terms: $S_n = \frac{a(1-r^n)}{a-r}$
\end{itemize}

\subsection{Basic Results}

\begin{itemize}
    \item \textit{annuity} -- payments made of regular intervals
    \item generally, all the payments are of the same magnitude
    \item annuity is generally a payment of 1 over $n$ periods
    \item we do have to find the equivalent rate of interest for the payment
        periods
    \item a payment plan for a general annuity
        \begin{center}
        \begin{tikzpicture}
            \def\f{1.5}
            \draw (0,0) -- (\f*7,0);
            \foreach \x in {\f*1,\f*2,\f*3,\f*4,\f*6}
              \draw (\x cm,3pt) -- (\x cm,-3pt);
            \draw (\f*1,0) node[below=3pt, align=center] {$0$}
                node[above=3pt] {$\ax{\angln}$};
            \draw (\f*2,0) node[below=3pt, align=center] {$1$}
                node[above=3pt] {$1$};
            \draw (\f*3,0) node[below=3pt, align=center] {$2$}
                node[above=3pt] {$1$};
            \draw (\f*4,0) node[below=3pt, align=center] {$3$}
                node[above=3pt] {$1$};
            \draw (\f*5,0) node[below=3pt, align=center] {$\dots$}
                node[above=3pt] {$ $};
            \draw (\f*6,0) node[below=3pt, align=center] {$n$}
                node[above=3pt] {$\sx{\angln}$};
        \end{tikzpicture}
        \end{center}
    \item present value of the annuity is $\ax{\angln}$
        \begin{equation}\nonumber
            \ax{\angln} = \frac{v(1-v^n)}{1-v} = \frac{1-v^n}{i}
        \end{equation}
    \item accumulated value of the annuity is $\sx{\angln}$
        \begin{equation}\nonumber
            \sx{\angln} = \ax{\angln}(1+i)^n = \frac{(1+i)^n-1}{i}
        \end{equation}
    \item to find actual value, we can multiply the present value with the
        actual value
    \item other symbols and values for annuities
        \begin{center}
        \begin{tikzpicture}
            \def\f{1.5}
            \draw (0,0) -- (\f*7,0);
            \foreach \x in {\f*1,\f*2,\f*3,\f*4,\f*6}
              \draw (\x cm,3pt) -- (\x cm,-3pt);
            \draw (\f*1,0) node[below=3pt, align=center] {$0$}
                node[above=3pt] {$\ax{\angln}$};
            \draw (\f*2,0) node[below=3pt, align=center] {$1$}
                node[above=3pt] {$\ax**{\angln}$};
            \draw (\f*3,0) node[below=3pt, align=center] {$2$}
                node[above=3pt] {$ $};
            \draw (\f*4,0) node[below=3pt, align=center] {$\dots$}
                node[above=3pt] {$ $};
            \draw (\f*5,0) node[below=3pt, align=center] {$n$}
                node[above=3pt] {$\sx{\angln}$};
            \draw (\f*6,0) node[below=3pt, align=center] {$n+1$}
                node[above=3pt] {$\sx**{\angln}$};
        \end{tikzpicture}
        \end{center}
    \item present value of the annuity described on the first payment
        $\ax**{\angln}$
        \begin{equation}\nonumber
            \ax**{\angln} = \frac{1-v^n}{d}
        \end{equation}
    \item accumulated value one period after the last payment $\sx**{\angln}$
        \begin{equation}\nonumber
            \sx**{\angln} = \frac{(1+i)^n-1}{d}
        \end{equation}
    \item we note two more identities
        \begin{equation}\nonumber
            \begin{aligned}
                \sx**{\angln} &= \ax**{\angln}(1+i)^n        \\
                1 &= d \cdot \ax**{\angln} + v^n
            \end{aligned}
        \end{equation}
\end{itemize}

\end{document}
