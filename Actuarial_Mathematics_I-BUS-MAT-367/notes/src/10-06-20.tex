\documentclass[../00_main.tex]{subfiles}

\begin{document}

\section{Class 06.10.2020}

\subsection{Annuities}

\begin{itemize}
    \item annuities can be viewed from many different angles with the same
        result
    \item \textit{annuity-immediate} are payments at the end of periods
    \item \textit{annuity-due} are payments made at the beginning of periods
\end{itemize}

\subsection{Perpetuities}

\begin{itemize}
    \item annuity whose payments continue forever
        \begin{equation}\nonumber
            \begin{aligned}
                \ax{\angl{\infty}} &= \lim_{n\rightarrow\infty} \ax{\angln}  \\
                &= \frac{1}{i}
            \end{aligned}
        \end{equation}
    \item we also have the perpetuity at the time of the first payment
        \begin{equation}\nonumber
            \begin{aligned}
                \ax**{\angl{\infty}} &= \ax{\angl{\infty}}(1+i) \\
                &= \frac{1}{d}
            \end{aligned}
        \end{equation}
\end{itemize}

\subsection{Unknown time and unknown rate of interest}

\begin{itemize}
    \item a fund of 5,000 will be used to award scholarships of 500 for as long
        as possible. If $i=0.09$, how many scholarships can be awarded?
        \begin{equation}\nonumber
            \begin{aligned}
                500 \cdot \ax{\angln} &\le 5,000 < 500 \cdot \ax{\angl{n+1}} \\
                \ax{\angln} &\le 10 < \ax{\angl{n+1}}   \\
                \ax{\angln} &= 10                       \\
                \frac{1-v^n}{i} &= 10                   \\
                n &= \frac{\ln(1-10i)}{\ln(v)}
            \end{aligned}
        \end{equation}
\end{itemize}

\end{document}
