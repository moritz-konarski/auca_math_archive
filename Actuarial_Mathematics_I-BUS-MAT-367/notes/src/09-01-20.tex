\documentclass[../00_main.tex]{subfiles}

\begin{document}

\section{Class 01.09.2020}

\subsection{General Introduction}

\begin{itemize}
    \item introduction of professor
    \item this is course 1 of 2
    \item quite popular elective course -- he is not sure why
    \item we'll dig deep inside the workings of insurance and the like
    \item it's a very well paid profession
    \item who is an actuary:
        \begin{itemize}
            \item completely independent professional
            \item actuarial association assigns audits to actuaries
            \item basically a type of auditor
            \item actuarial expertise is needed to make investments
        \end{itemize}
    \item this course will teach us the basics
    \item to become an actuary, you will have to pass 6 exams, we'll learn
        stuff for the first 2
    \item re-insurance: some of the richest companies in the world
    \item Parmenter is the main text book
    \item \textbf{course content}: chapters 1, 2, 3, and 4
\end{itemize}

\subsection{Accumulation Function}

\begin{itemize}
    \item the simplest financial transaction is an investment
    \item \textbf{principal}: initial investment
    \item \textbf{accumulated value}: total amount the money grows to
    \item \textbf{Amount Function}: amount of money at time $t$ from investment
        of the principal -- $A(t)$, $t$ is measured in years, $A(0)$ is the
        principal
    \item \textbf{Accumulation Function}: how much money increases as a percent
        value, where $a(0)=1$ (as there has been no change)
        \begin{equation}\nonumber
            a(t) = \frac{A(t)}{A(0)}
        \end{equation}
    \item accumulation functions can be any function where $a(0)=1$,
        additionally one would hope that it is increasing
    \item continuity is not required, depends on how interest is paid -- if
        fractional values of $t$ make sense it may be continuous, but if
        interest is paid discretely, it may be stepwise
    \item three types of accumulation functions
        \begin{enumerate}
            \item amount of interest earned each year is constant --
                linear graph, simple interest 
            \item the amount of interest increases over the years --
                exponential graph, compound interest
            \item if interest is paid out at fixed periods of time a piecewise
                function is used -- the amount of interest might be constant or
                increasing
        \end{enumerate}
    \item \textbf{Interest = Accumulated Value - Principal}
    \item to make this more practical, the \textit{effective rate of interest
        $i$} is used
    \item $i$ is the interest earned on a principal of 1 over the period of
        1 year -- amount of interest earned over 1 year divided by the value at
        the beginning of the year
        \begin{equation}\nonumber
            i = a(1) - 1
        \end{equation}
    \item $i$ can also be calculated with the amount function
        \begin{equation}\nonumber
            i = \frac{a(1)-a(0)}{a(0)} = \frac{A(1)-A(0)}{A(0)}
        \end{equation}
    \item $i$ can be calculated for the $n$th year by
        \begin{equation}\nonumber
            i = \frac{a(n)-a(n-1)}{a(n-1)} = \frac{A(n)-A(n-1)}{A(n-1)}
        \end{equation}
\end{itemize}

\subsection{Simple Interest}

\begin{itemize}
    \item primarily used between integer periods of time
    \item $a(t)$ is a straight line here -- the increase is linear
    \item general form of the equation is 
        \begin{equation}\nonumber
            a(t) = 1 + it
        \end{equation}
    \item interest earned each year is constant -- interest does not earn
        interest
    \item if the principal is $k$ at $t=0$
        \begin{equation}\nonumber
            A(t) = k(1+it)
        \end{equation}
    \item the effective rate of interest is not constant, it decreases over
        time
        \begin{equation}\nonumber
            i_n = \frac{i}{1 + i(n-1)}
        \end{equation}
    \item \textbf{exact simple interest}: count the last day, not the first
        \begin{equation}\nonumber
            t = \frac{\text{number of days}}{365}
        \end{equation}
    \item \textbf{ordinary simple interest (Banker's Rule)}: count the last 
        day, not the first
        \begin{equation}\nonumber
            t = \frac{\text{number of days}}{360}
        \end{equation}
    \item international markets use ordinary simple interest
\end{itemize}

\end{document}
