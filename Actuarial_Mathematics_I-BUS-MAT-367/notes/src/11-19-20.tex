\documentclass[../00_main.tex]{subfiles}

\begin{document}

\section{Class 11.19.2020}

\subsection{Price of a bond}

A bond is a certificate in which, in return for receiving an initial sum of 
money from the investor, the borrower agrees to pay interest at a specified 
rate (the coupon rate) until a specified date (the maturity date), and, at that 
time, to pay a fixed sum (the redemption value). The coupon rate is customarily 
quoted as a nominal rate convertible semiannually, and is applied to the face 
(or par) value, which is stated on the front of the bond. Usually the face and
redemption values are equal, but this is not always the case.
\begin{itemize}
    \item basically a different way to handle loans
    \item e.g. face amount of 500, redeemable at par in 10 years, coupon rate 
        11\% convertible semiannually $\rightarrow$ investor gets 20 payments
        of $(0.055)(500)$ and a lump sum payment of 500 at the end
    \item if one can buy the bond above for less than 500, their yield rate 
        will be higher, if they pay more, their yield rate will be lower
    \item $F$ -- face value or par of the bond
    \item $r$ -- coupon rate per interest period, the quoted rate is generally
        $2r$; each interest payment is $Fr$
    \item $C$ -- redemption value of the bond, often $D=F$
    \item $i$ -- yield rate per interest period
    \item $n$ -- number of interest periods until redemption date
    \item $P$ -- purchase price of the bond to get yield rate $i$
    \item a timeline of a standard bond
    \begin{center}
        \begin{tikzpicture}
            \def\f{1.5}
            \draw (0,0) -- (\f*6,0);
            \foreach \x in {\f*1,\f*2,\f*3,\f*5}
              \draw (\x cm,3pt) -- (\x cm,-3pt);
            \draw (\f*1,0) node[below=3pt, align=center] {$0$}
                node[above=3pt] {$P$};
            \draw (\f*2,0) node[below=3pt, align=center] {$1$}
                node[above=3pt] {$Fr$};
            \draw (\f*3,0) node[below=3pt, align=center] {$2$}
                node[above=3pt] {$Fr$};
            \draw (\f*4,0) node[below=3pt, align=center] {$\dots\dots$}
                node[above=3pt] {$ $};
            \draw (\f*5,0) node[below=3pt, align=center] {$n$}
                node[above=3pt, align=center] {$C$\\$Fr$};
        \end{tikzpicture}
    \end{center}
    \item formula for $P$
        \begin{equation}\nonumber
            P = (Fr)\ax{\angln i} + C(1+i)^{-n}
        \end{equation}
    \item formula for $i$ as an iterative method
        \begin{equation}\nonumber
            i = \frac{Fr(1-v^n)}{P - Cv^n}
        \end{equation}
    \item formula for $n$
        \begin{equation}\nonumber
            n = \frac{\ln \left( \frac{P - \frac{Fr}{i}}{C
                - \frac{Fr}{i}}\right)}{\ln v}
        \end{equation}
    \item if an investor buys a bond for less than the redemption value, he
        buys it at a discount -- the other way around he buys it at a premium
        ($P-C$)
\end{itemize}

\subsection{Book value}

\begin{itemize}
    \item the outstanding value of the bond at time $t$
    \item it is the present value of all future payments
    \item book value at time $t$ as the $t$th coupon has been paid
        \begin{equation}\nonumber
            B_t = (Fr) \ax{\angl{n-t}} + Cv^{n-t}
        \end{equation}
    \item the book value generally changes from $P$ at $t=0$ to $C$ at $t=n$
    \item also, for successive book values
        \begin{equation}\nonumber
            B_{t+1} = B_t(1+i) - Fr
        \end{equation}
    \item between book values we assume simple interest at rate $i$
    \item flat price of the bond is what we generally calculate -- the one that
        goes up and down
    \item market price (amortized value) is the interpolation between 
        successive book values -- we don't have the payment spikes in there
\end{itemize}

\subsection{Bond amortization schedules}

\begin{itemize}
    \item basically the same idea as with loan amortization schedules
    \item book value is in the last column of the table
    \item this will show how book value changes from $P$ to $C$
    \item book value at $t$ is $B_t$
    \item amount of coupon at $t+1$ is $Fr$
    \item amount of interest is $B_t i$
    \item change in book value is $Fr - B_t i$
    \item \textit{Example 1}: amount of interest and change in book value for
        the 15th coupon:
        \begin{itemize}
            \item interest: $B_t i = 950.83 \cdot 0.06 = 57.05$
            \item it's greater than the coupon value
            \item we modify the book value by the difference of new $-$ old, so
                we actually increase it
        \end{itemize}
    \item for a complete schedule, we start with finding the price using the
        main formula for it
    \item then we need to do the normal stuff
\end{itemize}

\subsection{Other topics}

\begin{itemize}
    \item \textit{Different interest periods}: we must convert one to the other
        in a sensible way -- i.e. coupons are the defining part
    \item \textit{Changing coupons}: apply the changing annuities stuff to the
        price formula to find the price at a certain yield rate
    \item \textit{Changing interest rates}: calculate the price using
        a piecewise function to find the present value of all the coupon
        payments at different rates and also the present value of the
        redemption value for the different rates
    \item \textit{Callable bonds}: borrower can redeem the bond anytime
        between the call date and the usual maturity date, no coupons are paid
        once the bond is redeemed. Sometimes different redemption values are
        paid for different redemption dates
    \item \textbf{important}: generally it is best to take the earliest value
        if $i < r$, else use the last value. If there is doubt, find all 
        possible values and compare
    \item \textit{Duration of a Bond}: $R_n$ are payments made at times $t_n$
        \begin{equation}\nonumber
            \bar d = \frac{\sum_{j=1}^n t_j v^{t_j} R_j}{\sum_{j=1}^n v^{t_j} R_j}
        \end{equation}
    \item the modified duration or volatility is 
        \begin{equation}\nonumber
            \bar v = \frac{\bar d}{1 + i}
        \end{equation}
\end{itemize}

\end{document}
