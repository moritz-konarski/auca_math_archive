\documentclass[../00_main.tex]{subfiles}

\begin{document}

\section{Class 10.09.2020}

\subsection{Nominal Rate of Interest}

\begin{itemize}
    \item $a(t) = (1+i)^t$ will be assumed in this section
    \item effective rates of interest can be given for any length of time
    \item to apply our previous formulae, we need to make sure that $t$ is the
        number of \textit{effective interest periods}
    \item generally, these periods are not years, but shorter periods
    \item a yearly rate or 12\% "convertible semiannually" actually means that
        you pay 6\% twice a year -- in this case it would actually be 12.36\%
    \item the effective interest rate increases the shorter the intervals
        between payments are
    \item the 12\% is a \textbf{nominal rate of interest}, meaning it is
        convertible over a period other than 1 year
    \item $i^{(m)}$ denotes the nominal rate of interest convertible $m$ times
        a year
        \begin{equation}\nonumber
            1+i = \left[ 1 + \frac{i^{(m)}}{m} \right]^m
        \end{equation}
    \item we can also define a nominal rate of discount $d^{(m)}$
        \begin{equation}\nonumber
            1-d = \left[ 1 - \frac{d^{(m)}}{m} \right]^m
        \end{equation}
    \item we also see that
        \begin{equation}\nonumber
            \left[ 1 + \frac{i^{(m)}}{m} \right]^m = 
                \left[ 1 - \frac{d^{(n)}}{n} \right]^{-n}
        \end{equation}
\end{itemize}

\subsection{Force of Interest}

\begin{itemize}
    \item our goal is to find nominal rates of interest that are equivalent to
        a certain effective annual rate of interest
    \item for example $i=0.12$ with the functions above gives the values
        \begin{center}
            \begin{tabular}{|c|c|c|c|c|c|}\hline
                $m$     & 1     & 2     & 5     & 10    & 50    \\\hline     
                $i^{(m)}$ & 0.12& 0.1166&0.1146 & 0.1140& 0.1135\\\hline
            \end{tabular}
        \end{center}
    \item we see that $i^{(m)}$ decreases as $m$ increases
    \item $m$ is approaching a limit, using L'Hopital's rule we can find it
        \begin{equation}\nonumber
            \begin{aligned}
                \delta &= \ln{(1+i)}        \\
                e^{\delta} &= 1 + i
            \end{aligned}
        \end{equation}
    \item $\delta$ is called the \textbf{force of interest}
    \item it represents the nominal rate of interest that is convertible
        \textit{continuously} -- serving as a good approximation of $i^{(m)}$
        for large $m$, like dayly conversions
    \item the second form of $\delta$ is useful because it makes conversions
        easier
    \item the derivative of $(1+i)^t$ by $t$ ($D$) can be rewritten to be
        \begin{equation}\nonumber
            \delta = \ln{(1+i)} = \frac{D[(1+i)^t]}{(1+i)^t} = 
                \frac{D[a(t)]}{a(t)}
        \end{equation}
    \item for compound interest $\delta = \ln{(1+i)}$, but for arbitrary
        accumulation functions it is
        \begin{equation}\nonumber
            \begin{aligned}
                \delta_t &= \frac{D[a(t)]}{a(t)}     \\
                \delta_t &= D[\ln{(a(t))}]
            \end{aligned}
        \end{equation}
    \item if $\delta_r$ is given and we want to find $a(t)$ we use
        \begin{equation}\nonumber
            a(t) = e^{\int_0^t{\delta_r}{dr}}
        \end{equation}
    \item we note that $i > \delta$
    \item the force of discount is the same as the force of interest
\end{itemize}

\end{document}
