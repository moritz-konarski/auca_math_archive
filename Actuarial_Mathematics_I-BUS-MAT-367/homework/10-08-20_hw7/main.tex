\documentclass[a4paper, 12pt, reqno]{amsart}

\newcommand{\titl}{Actuarial Mathematics Homework 7}

\usepackage{amssymb}
\usepackage{amsfonts}
\numberwithin{equation}{section}
\usepackage[margin=1in]{geometry}
\usepackage[english]{babel}
\usepackage[colorlinks, pdftitle={\titl},
    pdfauthor={Moritz M. Konarski}]{hyperref}
\usepackage{enumitem}
\usepackage{graphicx}
\usepackage{tikz}
\usetikzlibrary{snakes}
\renewcommand{\baselinestretch}{1.25}
\usepackage{actuarialsymbol}

\title{\titl}
\author{Moritz M. Konarski}
\date{\today}

\begin{document}

\maketitle

\section*{Parmenter Exercises 3--47 to 3--55}

\subsection*{3--47}

Loan of 6,000. Pay back 800 a year for as long as necessary, then a payment
less than 800 at the end. First payment due in 1 year and $i=0.11$ -- find the
number of payments and the smaller payment.
\begin{equation}\nonumber
    \begin{aligned}
        800 \cdot \ax{\angln} &\le 6,000 < 800 \cdot \ax{\angl{n+1}}    \\
        \ax{\angln} &\le 7.5 < \ax{\angl{n+1}}                          \\
        \ax{\angln} &= 7.5                                              \\
        \frac{1-v^n}{i} &= 7.5       \\
        n &= \frac{\ln(1-7.5i)}{\ln(v)} \\
        n &= 16.7015
    \end{aligned}
\end{equation}
We know that it takes 16 payments of 800 plus 1 smaller payment to pay off the
loan, 17 payments in total. The smaller payment is
\begin{equation}\nonumber
    \begin{aligned}
        800 \cdot \sx{\angl{16}} + X &= 6,000(1+i)^{16}     \\
        X &= 6,000(1+i)^{16} - 800 \cdot \sx{\angl{16}}         \\
        X &= 513.40721          \\
    \end{aligned}
\end{equation}

\subsection*{3--48}

Loan of 6,000. Pay back 70 a month for as long as necessary, then a payment
less than 70 at the end. First payment due in 1 month and $i=0.11$ per year -- 
find the number of payments and the smaller payment.
\begin{equation}\nonumber
    \begin{aligned}
        i &= \sqrt[12]{1.11} - 1
        70 \cdot \ax{\angln} &\le 6,000 < 70 \cdot \ax{\angl{n+1}}    \\
        \ax{\angln} &\le \frac{600}{7} < \ax{\angl{n+1}}            \\
        \ax{\angln} &= \frac{600}{7}                                  \\
        \frac{1-v^n}{i} &= \frac{600}{7}       \\
        n &= \frac{\ln(1-\frac{600}{7}i)}{\ln(v)} \\
        n &= 158.79946
    \end{aligned}
\end{equation}
We know that it takes 158 payments of 70 plus 1 smaller payment to pay off the
loan, 159 payments in total. The smaller payment is
\begin{equation}\nonumber
    \begin{aligned}
        70 \cdot \sx{\angl{158}} + X &= 6,000(1+i)^{158}         \\
        X &= 6,000(1+i)^{158} - 70 \cdot \sx{\angl{158}}         \\
        X &= 55.52574
    \end{aligned}
\end{equation}

\subsection*{3--49}

Loan of 6,000. Pay back 800 a year for as long as necessary, then a payment
less than 800 at the end. First payment due in 1 year and $i=0.11$ -- find the
number of payments and the smaller payment.
\begin{equation}\nonumber
    \begin{aligned}
        800 \cdot \ax{\angln} \cdot v &\le 6,000 < 800 \cdot
            \ax{\angl{n+1}}\cdot v \\
        \ax{\angln} \cdot v &\le 7.5 < \ax{\angl{n+1}} \cdot v         \\
        \ax{\angln} &= 7.5(1+i)                                        \\
        \frac{1-v^n}{i} &= 7.5(1+i)       \\
        n &= \frac{\ln(1-7.5i(1+i))}{\ln(v)} \\
        n &= 23.70608
    \end{aligned}
\end{equation}
We know that it takes 23 payments of 800 plus 1 smaller payment to pay off the
loan, 24 payments in total. The smaller payment is
\begin{equation}\nonumber
    \begin{aligned}
        800 \cdot \sx{\angl{23}} + X &= 6,000(1+i)^{24}         \\
        X &= 6,000(1+i)^{24} - 800 \cdot \sx{\angl{23}}         \\
        X &= 516.63265
    \end{aligned}
\end{equation}

\subsection*{3--50}

Fund of 5,000 is accumulated by $n$ annual payments of 50 by another $n$ 
payments of 100 plus a final payment of (as small as possible), one year after 
the final payment. At $i=0.08$ find $n$ and the final payment.
\begin{equation}\nonumber
    \begin{aligned}
        50 \cdot \sx{\angln}(1+i)^n + 100 \cdot \sx{\angln} &= 5,000    \\
        \sx{\angln}(1+i)^n + 2 \cdot \sx{\angln} &= 100 \\
        \frac{((1+i)^n)^2- (1+i)^n + 2(1+i)^n - 2}{i} &= 100 \\
        (1+i)^n &= c    \\
        c^2- c + 2c - 2 &= 100i \\
        0 &= c^2 + c - (2 + 100i) \\
        c_{1,2} &= \frac{-1 \pm \sqrt{41}}{2}   \\
        c_1 < 0 \quad &\quad c_2 = \frac{-1 + \sqrt{41}}{2} \\
        (1+i)^n &= \frac{-1 + \sqrt{41}}{2}     \\
        n &= \ln\left(\frac{-1 + \sqrt{41}}{2}\right) / \ln(1+i)     \\
        n &= 12.91342
    \end{aligned}
\end{equation}
We know that it takes 12 payments of 50 and 100 plus 1 smaller payment to
accumulate 5,000. The smaller payment is
\begin{equation}\nonumber
    \begin{aligned}
        50 \cdot \sx{\angl{12}}(1+i)^{121} + 100 \cdot \sx{\angl{12}} + X 
            &= 5,000       \\
        \sx{\angl{12}}(1+i)^{12} + 2 \cdot \sx{\angl{12}} + X 
            &= 100       \\
        X &= 100 - \sx{\angl{12}}(1+i)^{12} - 2 \cdot \sx{\angl{12}}    \\
        X &= 14.25811
    \end{aligned}
\end{equation}

\subsection*{3--51}

At what effective monthly rate will payments of 200 at the end of every month
for the next 3 years pay of 6500?
\begin{equation}\nonumber
    \begin{aligned}
        200 \cdot \ax{\angl{36}} &= 6500    \\
        \ax{\angl{36}} &= \frac{6500}{200}    \\
        \frac{1-v^{36}}{i} &= \frac{6500}{200}    \\
        0 &= \frac{6500}{200}i(1+i)^{36} - (1+i)^{36} + 1 \\
        \text{solving with a calculator}    \\
        i &= 0.03077
    \end{aligned}
\end{equation}

\subsection*{3--54}

A fund of 25,000 is to be accumulated at the end of 20 years by annual payments
of 500 at the end of each year. Find $i$
\begin{equation}\nonumber
    \begin{aligned}
        500 \cdot \sx{\angl{20}} &= 25,000    \\
        \sx{\angl{20}} &= \frac{25000}{500}    \\
        \frac{(1+i)^{20}-1}{i} &= 50    \\
        \text{solving with a calculator}    \\
        i &= 0.04878
    \end{aligned}
\end{equation}

\subsection*{3--55}

\begin{equation}\nonumber
    \begin{aligned}
        100 \cdot \sx{\angl{5}}(1+i)^5 + 200 \sx{\angl{5}} &= 2,200    \\
        \sx{\angl{5}}(1+i)^5 + 2 \sx{\angl{5}} &= 22    \\
        (1+i)^{10} + (1+i)^5 &= 22i \\
        \text{solving with a calculator}    \\
        i &= 0.66667
    \end{aligned}
\end{equation}

\end{document}
