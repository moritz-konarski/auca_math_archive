\documentclass[a4paper, 12pt, reqno]{amsart}

\newcommand{\titl}{Actuarial Mathematics Homework 10}

\usepackage{amssymb}
\usepackage{amsfonts}
\numberwithin{equation}{section}
\usepackage[margin=1in]{geometry}
\usepackage[english]{babel}
\usepackage[colorlinks, pdftitle={\titl},
    pdfauthor={Moritz M. Konarski}]{hyperref}
\usepackage{enumitem}
\usepackage{graphicx}
\usepackage{tikz}
\usetikzlibrary{snakes}
\renewcommand{\baselinestretch}{1.25}
\usepackage{actuarialsymbol}

\title{\titl}
\author{Moritz M. Konarski}
\date{\today}

\begin{document}

\maketitle

\section*{Parmenter Exercises 4--26 to 4--37}

\subsection*{4--26}

Loan of 10,000 taken out on March 1, 1995 at 8\% p.a. Interest is paid
annually. To repay on March 1, 2002 a sinking fund is established on March 1,
1996; annual payments; $i=0.09$.
\begin{enumerate}[label=(alph*)]
    \item amount of each sinking fund payment
        \begin{equation}\nonumber
            \begin{gathered}
                10,000(1+0.08) = X \ax{\angl{6}0.09}     \\
                10,000(1.08) = X \ax{\angl{6}0.09}     \\
                X = \frac{10,800}{\ax{\angl{6}0.09}}         \\
                X = 2,407.53366
            \end{gathered}
        \end{equation}
    \item total amount the borrower must pay p.a.
        \begin{equation}\nonumber
            \begin{gathered}
                X = 10,000 \cdot 0.08 + 2,407.53366  \\
                X = 3207.53366
            \end{gathered}
        \end{equation}
    \item total amount the borrower must pay p.a.
        \begin{equation}\nonumber
            \begin{gathered}
                10,000 \cdot i = 3207.53366     \\
                i = 0.32075
            \end{gathered}
        \end{equation}
\end{enumerate}

\subsection*{4--27}

50,000 at 11\% p.a. Repayment through amortization, 20 equal payments at the
end of each year. After the 10th payment we do a sinking fund for the now 12\%
loan, fund earns 14\%, we continue with 10 annual payments. Find (a) the
sinking fund deposit and (b) compare the new total payment to the old one.
\begin{enumerate}[label=(alph*)]
    \item sinking fund deposit
        \begin{equation}\nonumber
            \begin{gathered}
                OP_{10} = 50,000(1+i)^{10} - X \cdot \sx{\angl{10}} \\
                X = \frac{50,000}{\ax{\angl{20}}}        \\
                OP_{10} = 50,000(1+0.11)^{10} - \frac{50,000}{\ax{\angl{20}0.11}}
                    \cdot \sx{\angl{10}0.11}            \\
                OP_{10} = 36977.20303                   \\
                D = \frac{OP_{10}}{\sx{angl{10}0.14}}    \\
                D = 1912.2221
            \end{gathered}
        \end{equation}
    \item compare old and new total payments
        \begin{equation}\nonumber
            \begin{gathered}
                OTP = 20 \cdot X = 125,575.63688     \\
                NTP = 10 \cdot X + 10 \cdot 0.12 \cdot OP_{10} + 10 \cdot D 
                    = 126,282.68306
            \end{gathered}
        \end{equation}
    The new total payment is slightly larger.
\end{enumerate}

\subsection*{4--28}

Repay a loan of 1,000 with a sinking fund over 10 years. Interest on the loan
is $i$, on the sinking fund $i'$. The total payment each year is 125.
\begin{enumerate}[label=(alph*)]
    \item if the interest rates are equal, find $i$
        \begin{equation}\nonumber
            \begin{gathered}
                1,000 = FP \cdot \sx{\angl{10}i'} \\
                FP = \frac{1,000}{\sx{\angl{10}i'}}
                IP = 1,000 \cdot i              \\
                125 = IP + FP                   \\
                125 = 1,000i + \frac{1,000}{\sx{\angl{10}i'}}       \\
                125 = 1,000i + \frac{1,000}{\sx{\angl{10}i}}        \\
                0.125 = i + \frac{i}{(1+i)^{10}-1}                  \\
            \end{gathered}
        \end{equation}
        I couldn't solve this any further.
    \item Because the borrower still needs to deposit something into the
        sinking fund each month, the maximum payment towards interest can be
        124.99 at most. This represents an interest rate of 12.499\%.
\end{enumerate}

\subsection*{4--29}

\begin{enumerate}[label=(alph*)]
    \item the payment on interest is 115
    \item $149.06 - 115 = 34.06$
    \item $34.06 \cdot \sx{\angl{8}} = 375.62982$
\end{enumerate}

\subsection*{4--30}

\begin{enumerate}[label=(alph*)]
    \item $5000*(1.09)^{10} \cdot (1.11)^{10} - 5,000 \cdot 0.09
        \cdot \sx{\angl{10}0.09} \cdot (1.11)^10 - 5,000 \cdot 0.11
        \cdot \sx{\angl{10}0.11}$ \\ $4999.9999999999945$
    \item $i=0.09$
    \item $565.12427$
    \item payment is first 537.29812 and then 637.29812
\end{enumerate}

\subsection*{4--31}

\begin{enumerate}[label=(alph*)]
    \item $556.50394$
    \item $i = 18.55\%$
\end{enumerate}

\subsection*{4--32}

\begin{enumerate}[label=(alph*)]
    \item 
    \item 
\end{enumerate}

\end{document}
