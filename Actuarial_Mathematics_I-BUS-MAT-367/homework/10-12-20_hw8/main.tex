\documentclass[a4paper, 12pt, reqno]{amsart}

\newcommand{\titl}{Actuarial Mathematics Homework 8}

\usepackage{amssymb}
\usepackage{amsfonts}
\numberwithin{equation}{section}
\usepackage[margin=1in]{geometry}
\usepackage[english]{babel}
\usepackage[colorlinks, pdftitle={\titl},
    pdfauthor={Moritz M. Konarski}]{hyperref}
\usepackage{enumitem}
\usepackage{graphicx}
\usepackage{tikz}
\usetikzlibrary{snakes}
\renewcommand{\baselinestretch}{1.25}
\usepackage{actuarialsymbol}

\title{\titl}
\author{Moritz M. Konarski}
\date{\today}

\begin{document}

\maketitle

\section*{Parmenter Exercises 3--60 to 3--66}

\subsection*{3--60}

Money is borrowed, extracted in 5 annual payments $X$. Repayment starts 1 year
after the last extraction. It is repaid in 20 annual payments, first 100, then 
200, etc. If $i=0.132$, find $X$.
\begin{equation}\nonumber
    \begin{gathered}
        X\sx{\angl{5}} = 100\ax{\angl{20}} + 100\left[\frac{\ax{\angl{20}
            - 20 v^{20}}}{i}\right] \\
        X = \frac{100\ax{\angl{20}} + 100\left[ \frac{\ax{\angl{20}
            - 20 v^{20}}}{i}\right]}{\sx{\angl{5}}}\\
        X = 719.85057
    \end{gathered}
\end{equation}

\subsection*{3--61}

I could not solve this one.

\subsection*{3--63}

Present value of a perpetuity. After 1 year 100 is paid, then 200, etc. When
1500 is paid, the amount remains constant forever. 
\begin{equation}\nonumber
    \begin{gathered}
        A = P\ax{\angl{15}} + Q \left[ \frac{\ax{\angl{15}} - 15 v^{15}}{i} 
            \right] + (\ax{\angl{\infty}} - \ax{\angl{15}}) \cdot 1,500//
        A = 100\ax{\angl{15}} + 100 \left[ \frac{\ax{\angl{15}} - 15 v^{15}}{i} 
            \right] + (\ax{\angl{\infty}} - \ax{\angl{15}}) \cdot 1,500 \\
        A = 12,652.20497
    \end{gathered}
\end{equation}

\subsection*{3--64}

Annuity for 20 years, first payment immediately, $i=0.11$, payments increase by
10\% each year, starting with 1,000. We can find the geometric sum.
\begin{equation}\nonumber
    \begin{gathered}
        \ax**{\angl{20}} = \frac{S_{20}}{(1+i)^{19}} \\
        \ax**{\angl{20}} = \frac{1,000(1.1)^{20} \left( \frac{1-r^20}{1-r} 
            \right)}{(1+i)^{19}} \\
        \ax**{\angl{20}} = \frac{1,000(1.1)^{20} \left( \frac{1-\frac{1.1}
            {1.11}^20}{1-\frac{1.1}{1.11}} 
            \right)}{(1+i)^{19}} \\
        \ax**{\angl{20}} = 20,215.10754
    \end{gathered}
\end{equation}

\subsection*{3--66}

Annuity for 20 years, first payment immediately, $i=0.09$, payments are 1, 4,
9, 16, ..., starting immediately.
\begin{equation}\nonumber
    \begin{gathered}
        \ax**{\angl{20}} = \frac{\sum_{j=1}^{20} j^2 (1
            + i)^{20-j}}{(1+i)^{19}}\\
        \ax**{\angl{20}} = 887.15787
    \end{gathered}
\end{equation}

\end{document}
