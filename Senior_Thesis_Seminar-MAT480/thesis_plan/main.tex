\documentclass[a4paper, 12pt, reqno]{article}

\newcommand{\titl}{Senior Thesis Research Project Plan}
\newcommand{\auth}{Moritz M. Konarski, AMI--117}
\usepackage{amsmath}
\usepackage{amssymb}
\usepackage[margin=1in]{geometry}
\usepackage[english]{babel}
\usepackage[pdftitle={\titl}, pdfauthor={\auth}, final]{hyperref}
\usepackage{mathptmx}
\usepackage[T1]{fontenc}
\usepackage[document]{ragged2e}
%\usepackage{draftwatermark}
%\SetWatermarkScale{1}
%\SetWatermarkColor[gray]{0.9}
\setlength{\RaggedRightParindent}{0.25in}

\begin{document}
\begin{center}
    \begin{large}
        \textbf{\titl{}}\\
    \end{large}
        \vspace{4pt}
        \auth{}\\
        \vspace{3pt}
        Supervised by Professor Taalaibek Imanaliev\\
        \vspace{3pt}
        \today{}
\end{center}

\subsubsection*{Title}

Mathematical Modeling of ECG Abnormalities

\subsubsection*{Goal}

Develop a simple, accurate, and automatized electrocardiographic analysis tool 
for the diagnosis of various forms of ischemic heart disease.

\subsubsection*{Major Tasks}
\begin{itemize}
    \item Scanning of paper--based ECGs of normal and abnormal heart beats
    \item Approximation of the obtained data by an appropriate method
        \begin{itemize}
            \item Determine an appropriate method (Fourier Series, Fast
                Fourier Series, or similar)
            \item Perform analysis of existence, uniqueness of solution, error
                estimates
        \end{itemize}
    \item Creation of a comparison algorithm for normal and abnormal ECGs
        \begin{itemize}
            \item Account for age, gender, etc
            \item Use time and spacial comparisons, possibly almost--periodic
                functions
        \end{itemize}
    \item Create a sufficient digital database of diagnosed ECGs
    \item Make automated diagnosis that may assist with a doctor's diagnosis
\end{itemize}

\subsubsection*{Plan}
\begin{enumerate}
    \item Literature analysis
    \item Selection of proper software
    \item Collecting ECGs
    \item Scan paper-based ECGs
    \item Analysis of digital ECGs
    \item Find criteria for ECG abnormalities
    \item Automate ECG comparison
    \item Create database of ECG abnormalities
    \item Develop software for automatic diagnoses
    \item Analyse the resulting statistical data
    \item Write and publish a scientific paper
\end{enumerate}

\nocite{*}
\renewcommand{\refname}{\normalsize References}
\small
\bibliography{../bibliography}
\bibliographystyle{ieeetr}

\end{document}
